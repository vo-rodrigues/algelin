\section{Resolução de Sistemas Homogêneos}
Com já o básico operacional de matrizes desenvolvido, ganhamos algumas ferramentas para estudar sistemas lineares.

\begin{proposition}
    Seja $\mathbb K$ um corpo e $A\in M_{m \times n}(\mathbb K)$ uma matriz. Se $B \in M_{m}(\mathbb K)$ é uma matriz invertível, então o sistema linear homogêneo $AX=0$ é equivalente ao sistema linear homogêneo $BAX=0$.
\end{proposition}
\begin{proof}
    Já vimos que toda solução do sistema linear homogêneo $AX=0$ é também solução do sistema linear homogêneo $BAX=0$.
    
    Sendo $B^{-1}$ a inversa de $B$, segue que toda solução do sistema linear homogêneo $BAX=0$ é também solução do sistema linear homogêneo $B^{-1}BAX=0$.
    Como $B^{-1}BA=A$, temos que $B^{-1}BAX=0$ é equivalente a $AX=0$.
\end{proof}

Agora iniciaremos o estudo direto da resolução de sistemas lineares homogêneos.
Mais tarde, veremos como resolver sistemas lineares não homogêneos a partir das técnicas que desenvolveremos aqui.

Voltemos a nossa discussão intuitiva.
Considere o seguinte sistema linear homogêneo:
\begin{equation*}
    \begin{cases}
        2x_1 + 3x_2 = 0 \\
        x_1 - x_2 = 0
    \end{cases}
\end{equation*}

Sistemas homogêneos sempre tem a chamada \emph{solução trivial}: a atribuição do valor $0$ a cada variável. Será que o sistema acima possui outras soluções? 

Vamos considerar três operações que podemos realizar nesse sistema:

A primeira: multiplicar alguma das linhas por um número não nulo.
É imediato que, por exemplo, ao multiplicar a primeira linha por $\frac{1}{2}$, obtemos um sistema equivalente.
\begin{equation*}
    \begin{cases}
        x_1 + \frac{3}{2}x_2 = 0 \\
        x_1 - x_2 = 0
    \end{cases}
\end{equation*}
A segunda: trocar linhas de posição. É imediato que o sistema acima é equivalente ao seguinte:
\begin{equation*}
    \begin{cases}
        x_1 - x_2 = 0 \\
        x_1 + \frac{3}{2}x_2 = 0
    \end{cases}
\end{equation*}

E a terceira: trocar uma linha por ela própria, mais um múltiplo de outra.
Por exemplo, o sistema acima é equivalente ao seguinte:

\begin{equation*}
    \begin{cases}
        3x_1 + 2x_2 = 0 \\
        x_1 + \frac{3}{2}x_2 = 0
    \end{cases}
\end{equation*}

Para se obter esse sistema, somamos, na primeira linha, o dobro da segunda.
Para obter novamente a primeira linha original a partir do novo sistema, basta subtrair, da nova primeira linha, o dobro da segunda.
Logo, os sistemas são equivalentes.

Nossa afirmação é que, utilizando sistematicamente essas três operações, que chamaremos de \emph{operações elementares nas linhas de um sistema linear}, podemos resolver completamente qualquer sistema linear.
No nosso exemplo, olhando novamente para o sistema original, e somando na primeira linha o oposto da segunda (o produto da segunda por $-1$), obtemos o seguinte sistema:

\begin{equation*}
    \begin{cases}
        0x_1 + \frac{5}{2}x_2 = 0 \\
        x_1 - x_2 = 0
    \end{cases}
\end{equation*}
Multiplicando a primeira linha por $\frac{2}{5}$ e trocando as linhas de posição, obtemos o seguinte sistema:
\begin{equation*}
    \begin{cases}
        x_1 - x_2 = 0 \\
        0x_1 + x_2 = 0
    \end{cases}
\end{equation*}
Finalmente, somando na primeira linha, a segunda, obtemos o seguinte sistema:
\begin{equation*}
    \begin{cases}
        x_1 = 0 \\
        x_2 = 0
    \end{cases}
\end{equation*}

O que mostra que a única solução do sistema original é a solução trivial.

Vamos, nesta seção, sistematizar e formalizar este raciocínio a partir da linguagem de matrizes.

\begin{definition}
    Sejam $m, n$ inteiros positivos.
    Chamamos de \emph{operações elementares em linhas} em $M_{m \times n}(\mathbb K)$ as seguintes operações.
        \begin{enumerate}
            \item Para algum $l\in \{1, \ldots, m\}$ e $\lambda \in \mathbb K$ não nulo, $e_{l\lambda}^1((a_{ij})_{i, j})=(b_{ij})_{i, j}$ onde:
            \begin{equation*}
                b_{ij} = \begin{cases}
                    \lambda a_{ij} & \text{se } i=l \\
                    a_{ij} & \text{caso contrário.}
                \end{cases}
            \end{equation*}
            \item Para $l, l'\in \{1, \ldots, m\}$, $e_{l l'}^2((a_{ij})_{i, j})=(b_{ij})_{i, j}$ onde:
            \begin{equation*}
                b_{ij} = \begin{cases}
                    a_{l'j} & \text{se } i=l \\
                    a_{lj} & \text{se } i=l' \\
                    a_{ij} & \text{caso contrário.}
                \end{cases}
            \end{equation*}
            \item Para $l, l'\in \{1, \ldots, m\}$ e $\lambda \in \mathbb K$, $e_{l l'\lambda}^3((a_{ij})_{i, j})=(b_{ij})_{i, j}$ onde:
            \item \begin{equation*}
                b_{ij} = \begin{cases}
                    a_{lj} + \lambda a_{l'j} & \text{se } i=l \\
                    a_{ij} & \text{caso contrário.}
                \end{cases}
            \end{equation*}
        \end{enumerate}
\end{definition}

Tais operações elementares correspondem a produtos por certas matrizes.

\begin{definition}
    Seja $m$ um inteiro positivo.
    Chamamos de \emph{matrizes elementares} de $M_{m}(\mathbb K)$ as seguintes matrizes.
        \begin{enumerate}
            \item Para algum $l\in \{1, \ldots, m\}$ e $\lambda \in \mathbb K$ não nulo, $E_{l\lambda}^1=(b_{ij})_{i, j}$, onde:
            \begin{equation*}
                b_{ij} = \begin{cases}
                    \lambda & \text{se } i=j=l \\
                    1 & \text{se } i=j\neq l \\
                    0 & \text{caso contrário.}
            \end{cases}
            \end{equation*}
            \item Para $l, l'\in \{1, \ldots, m\}$, $E_{l l'}^2=(b_{ij})_{i, j}$, onde:
            \begin{equation*}
                b_{ij} = \begin{cases}
                    1 & \text{se } i=j\notin\{l, l'\} \\
                    1 & \text{se } i=l, j=l' \\
                    1 & \text{se } i=l', j=l \\
                    0 & \text{caso contrário.}
                \end{cases}
            \end{equation*}
            \item Para $l, l'\in \{1, \ldots, m\}$ e $\lambda \in \mathbb K$, $E_{l l'\lambda}^3=(b_{ij})_{i, j}$, onde:
            \begin{equation*}
                b_{ij} = \begin{cases}
                    1 & \text{se } i=j \\
                    \lambda & \text{se } i=l, j=l' \\
                    0 & \text{caso contrário.}
                \end{cases}
            \end{equation*}
        \end{enumerate}

        \begin{lemma}
            Sejam $m, n$ inteiros positivos e $A\in M_{m \times n}(\mathbb K)$.
            Então:
            \begin{enumerate}
                \item Dado $l\in \{1, \ldots, m\}$ e $\lambda \in \mathbb K$ não nulo, $E_{l\lambda}^1 A = e_{l\lambda}^1(A)$.
                \item Dado $l, l'\in \{1, \ldots, m\}$, $E_{l l'}^2 A = e_{l l'}^2(A)$.
                \item Dado $l, l'\in \{1, \ldots, m\}$ e $\lambda \in \mathbb K$, $E_{l l'\lambda}^3 A = e_{l l'\lambda}^3(A)$.
            \end{enumerate}
        \end{lemma}

        \begin{proof}
            Sejam $i, j$ tais que $1\leq i \leq m$ e $1\leq j \leq n$.
            Escreva $A=(a_{ij})_{i, j}$ e, em cada caso, seja $(b_{ij})_{i, j}$ a matriz elementar em questão.
            \begin{enumerate}
                \item Se $i\neq l$, o elemento $ij$ de $E_{l\lambda}^1 A$ é dado por $\sum_{k=1}^n b_{ik}a_{kj}=a_{ij}$.
                Já se $i=l$, o elemento $ij$ de $E_{l\lambda}^1 A$ é dado por $\sum_{k=1}^n b_{ik}a_{kj}=\lambda \lambda a_{lj}$.
                Assim, $E_{l\lambda}^1 A = e_{l\lambda}^1(A)$.
                \item Se $i\notin\{l, l'\}$, o elemento $ij$ de $E_{l l'}^2 A$ é dado por $\sum_{k=1}^n b_{ik}a_{kj}=a_{ij}$.
                Se $i=l$, o elemento $ij$ de $E_{l l'}^2 A$ é dado por $\sum_{k=1}^n b_{lk}a_{kj}=a_{l'j}$.
                Finalmente, se $i=l'$, o elemento $ij$ de $E_{l l'}^2 A$ é dado por $\sum_{k=1}^n b_{l'k}a_{kj}=a_{lj}$.
                Assim, $E_{l l'}^2 A = e_{l l'}^2(A)$.
                \item Se $i\neq l$, o elemento $ij$ de $E_{l l'\lambda}^3 A$ é dado por $\sum_{k=1}^n b_{ik}a_{kj}=a_{ij}$.
                Se $i=l$, o elemento $ij$ de $E_{l l'\lambda}^3 A$ é dado por $\sum_{k=1}^n b_{lk}a_{kj}=a_{ij}+\lambda a_{l'j}$.
                Assim, $E_{l l'\lambda}^3 A = e_{l l'\lambda}^3(A)$.
            \end{enumerate}
        \end{proof}
\end{definition}

\begin{lemma}
    Seja $m$ um inteiro positivo e $\mathbb K$ um corpo. As matrizes elementares de $M_{m}(\mathbb K)$ são invertíveis.
\end{lemma}
\begin{proof}
    Consideremos $E_{l\lambda}^1$, onde $l\in \{1, \ldots, m\}$ e $\lambda \in \mathbb K$ não nulo. É imediato que $I_n=e_{l\lambda^{-1}}^1(E_{l\lambda}^1)$, de modo que $I_n=E_{l\lambda^{-1}}^1 E_{l\lambda}^1$.
    Da mesma forma, $I_n=E_{l\lambda}^1 E_{l\lambda^{-1}}^1$

    Agora consideremos $E_{l l'}^2$, onde $l, l'\in \{1, \ldots, m\}$. É imediato que $I_n=e_{l l'}^2(E_{l l'}^2)$, de modo que $I_n=E_{l l'}^2 E_{l l'}^2$.

    Finalmente, consideremos $E_{l l'\lambda}^3$, onde $l, l'\in \{1, \ldots, m\}$ e $\lambda \in \mathbb K$. É imediato que $I_n=e_{l l'(-\lambda)}^3(E_{l l'\lambda}^3)$, de modo que $I_n=E_{l l'(-\lambda)}^3 E_{l l'\lambda}^3$.
    Da mesma forma, $I_n=E_{l l'\lambda}^3 E_{l l'(-\lambda)}^3$.
\end{proof}

\begin{corollary}
    Sejam $A, B \in M_{m \times n}(\mathbb K)$ matrizes tais que $A$ é obtida a partir de $B$ aplicando-se sucessivas operações elementares em linhas.
    Então os sistemas lineares homogêneos $AX=0$ e $BX=0$ são equivalentes.
\end{corollary}

Estudemos agora o seguinte sistema:

\begin{equation}\label{eqn:sistemaExemplo1}
    \begin{cases}
        2x_1 + 6x_2 + 4x_3 + x_4 = 0 \\
        3x_1 - 3x_2 - 4x_3 + x_4 = 0
    \end{cases}
\end{equation}

Vamos resolvê-lo utilizando operações elementares em linhas.
A matriz associada a esse sistema é:
\begin{equation*}
    A = \begin{pmatrix}
        2 & 6 & 4 & 1 \\
        3 & -3 & -4 & 1
    \end{pmatrix}
\end{equation*}
Multiplicando a primeira linha por $\frac{1}{2}$, obtemos:
\begin{equation*}
    A_1 = \begin{pmatrix}
        1 & 3 & 2 & \frac{1}{2} \\
        3 & -3 & -4 & 1
    \end{pmatrix}
\end{equation*}
Agora, subtraindo da segunda linha o triplo da primeira, obtemos:
\begin{equation*}
    A_2 = \begin{pmatrix}
        1 & 3 & 2 & \frac{1}{2} \\
        0 & -15 & -12 & -\frac{3}{2}
    \end{pmatrix}
\end{equation*}
Multiplicando a segunda linha por $-\frac{1}{15}$, obtemos:
\begin{equation*}
    A_3 = \begin{pmatrix}
        1 & 3 & 2 & \frac{1}{2} \\
        0 & 1 & \frac{4}{5} & \frac{1}{10}
    \end{pmatrix}
\end{equation*}
Finalmente, subtraindo da primeira linha o triplo da segunda, obtemos:
\begin{equation*}
    A_4 = \begin{pmatrix}
        1 & 0 & -\frac{2}{5} & \frac{1}{5} \\
        0 & 1 & \frac{4}{5} & \frac{1}{10}
    \end{pmatrix}
\end{equation*}
Assim, o sistema da Equação~\eqref{eqn:sistemaExemplo1} é equivalente ao seguinte sistema:
\begin{equation}\label{eqn:sistemaExemplo2}
    \begin{cases}
        x_1 - \frac{2}{5}x_3 + \frac{1}{5}x_4 = 0 \\
        x_2 + \frac{4}{5}x_3 + \frac{1}{10}x_4 = 0
    \end{cases}
\end{equation}
Ou, de forma mais simples:
\begin{equation*}
    \begin{cases}
        x_1 = \frac{2}{5}x_3 - \frac{1}{5}x_4 \\
        x_2 = -\frac{4}{5}x_3 - \frac{1}{10}x_4
    \end{cases}
\end{equation*}
Escolhidos quaisquer valores reais de $x_3$ e $x_4$, e utilizando-se as expressões acima para $x_1, x_2$, obtemos uma solução do sistema original, e todas as soluções do sistema original são obtidas dessa forma.
Em outras palavras, o conjunto solução do sistema da Equação~\eqref{eqn:sistemaExemplo1} é dado por:
\begin{equation*}
    \left\{\left(\frac{2}{5}x_3 - \frac{1}{5}x_4, -\frac{4}{5}x_3 - \frac{1}{10}x_4, x_3, x_4\right) \mid x_3, x_4 \in \mathbb R\right\}
\end{equation*}

Nosso objetivo é mostrar que esse método é geral.
Notemos que o sistema da Equação~\eqref{eqn:sistemaExemplo2} é notável: basta observá-lo para, de imediato, escrever seu conjunto solução.

Definiremos, abaixo, uma classe de matrizes semelhantes a essa, e, a seguir, mostraremos que para qualquer matriz, existe uma sequência finita de operações elementares em linhas que a transforma em uma matriz dessa classe.
A demonstração desse teorema será construtiva, de modo que ela nos dará o algoritmo necessário para resolver qualquer sistema linear homogêneo.

\begin{definition}
    Seja $m, n$ inteiros positivos e $\mathbb K$ um corpo.

    Seja $A \in M_{m \times n}(\mathbb K)$ e $i$ entre $1$ e $m$.

    Dizemos que a $i$-ésima linha de $A$ possui pivô se ela é não nula e $j \in \{1, \ldots, n\}$ tal que $a_{ij}=1$ e $a_{i'j}=0$ para todo $i'\neq i$, e, além disso, $a_{ij}$ é o primeiro elemento não nulo da linha.
    Nesse caso, o elemento $a_{ij}$ é chamado de \emph{pivô} da $i$-ésima linha de $A$, e $j$ é a \emph{posição do pivô} da $i$-ésima linha de $A$.
    

    Dizemos que uma matriz $A\in M_{m \times n}(\mathbb K)$ está na \emph{forma escalonada} se, para todo $i, i' \in \{1, \ldots, m\}$ com $i<i'$, valem as seguintes condições.
    \begin{enumerate}[label=\alph*)]
        \item Se a $i$-ésima linha de $A$ é nula, então a $i'$-ésima linha de $A$ também é nula.
        \item Se a $i$-ésima linha de $A$ não é nula, então ela possui um pivô.
        \item Caso as linhas $i, i'$ de $A$ não são nulas, e $j_i, j_{i'}$ são as posições dos pivôs dessas linhas, então $j_i < j_{i'}$.
    \end{enumerate}

    Para $i \in \{1, \ldots, m\}$ com $i\leq S$, a posição $(i, j_i)$ é dita ser um \emph{pivô} de $A$.
\end{definition}
Observe que a matriz identidade $I_n$ e a matriz nula estão ambas na forma escalonada.

Como ocorre na discussão, acima, temos:

\begin{theorem}
    Seja $m, n$ inteiros positivos e $\mathbb K$ um corpo.
    Então, para toda matriz $A\in M_{m \times n}(\mathbb K)$ na forma escalonada as soluções do sistema $AX=0$ são dadas, segundo a notação acima, pelas $n$-uplas
    $(x_j: 1\leq j\leq n)$, onde $x_j$ é qualquer se para nenhum $i$, $(i, j)$ é um pivô de $A$.
    Já se $(i, j)$ é um pivô de $A$, então $x_j=\sum\left(-a_{ij'}x_{j'}: j<j',\, j' \text{ não é pivô de } A\right)$.
\end{theorem}

\begin{theorem}
    Seja $m, n$ inteiros positivos e $\mathbb K$ um corpo.
    Então, para toda matriz $A\in M_{m \times n}(\mathbb K)$, existe uma sequência finita de operações elementares em linhas que transforma $A$ em uma matriz na forma escalonada.
\end{theorem}
\begin{proof}
    Provaremos por indução em $n$.

    Se $n=1$, então $A$ é uma matriz coluna.
    Se $A$ é nula, então $A$ já está na forma escalonada.

    Caso contrário, trocando duas linhas de posição, obtemos uma matriz coluna cujo primeiro elemento é não nulo.
    Multiplicando a primeira linha pelo inverso de seu novo elemento, obtemos uma matriz coluna cujo primeiro elemento é $1$.
    Finalmente, somando em cada outra linha $i$ não nula o múltiplo da primeira linha pelo oposto do elemento na posição $i1$, obtemos a matriz escalonada abaixo:

    \begin{equation*}
        \begin{pmatrix}
            1 \\
            0 \\
            0 \\
            \vdots \\
            0
        \end{pmatrix}
    \end{equation*}

Para a hipótese de indução, suponha que toda matriz $A \in M_{m \times n}(\mathbb K)$ pode ser escalonada por uma sequência finita de operações elementares em linhas.

Se $A \in M_{m \times (n+1)}(\mathbb K)$ e $B \in M_{m \times n}(\mathbb K)$ é a matriz obtida a partir de $A$ eliminando-se a última coluna, então, por hipótese de indução, existe uma sequência finita de operações elementares em linhas que transforma $B$ em uma matriz na forma escalonada.
Perceba que se $e$ é uma operação elementar em linhas e $\pi$ é a operação que elimina de uma matriz sua última coluna, então $e\circ \pi(A)=\pi\circ e(A)$.
Logo, sendo $e_1, \dots e_l$ operações elementares em linhas tais que $e_1\circ \cdots \circ e_l(B)$ é a matriz na forma escalonada, temos que $\pi\circ e_1\circ \cdots \circ e_l(A)=e_1\circ \cdots \circ e_l\circ \pi(A)=e_1\circ \cdots \circ e_l(B)$ está na forma escalonada.

Ou seja, aplicando-se em $A$ a sequência de operações elementares em linhas $e_1, \dots e_l$ que escalona $B$, obtemos uma matriz $C$ que, a menos de sua última coluna, está na forma escalonada, $S$.

Se $A'$ está já na forma escalonada, não há nada para fazer. Caso contrário:

\begin{itemize}
    \item Caso 1: existem linhas $i<i'$ de $C$ tais que $C_{i'}$ não é nula e $C_i$ é nula. Como $S$ está na forma escalonada, o único elemento não nulo de $C_{i'}$ está na última coluna.
    Podemos multiplicá-lo pelo seu inverso e, como no caso base, utilizar esta linha para zerar as outras entradas da última coluna da matriz, e, em seguida, reordená-la para transformá-la em uma matriz na forma escalonada.
    \item Caso 2: alguma linha não nula não possui pivô. Notemos que cada pivô de $S$ é um pivô de $C$.
    Seja $c_{ij}$ o primeiro elemento não nulo de $C$ que não é pivô.

    Se, por absurdo $j\leq n$, $c_{ij}$ não é pivô de $S$, logo, existe $j'<j$ tal que $c_{ij'}$ é pivô de $S$.
    Assim, a $i$-ésima linha de $S$ possui pivô, e, portanto, a $i$-ésima linha de $C$ também possui pivô, o que é um absurdo.
    Logo, o único elemento não nulo de $C_i$ está na última coluna e agimos como no caso anterior.
    \item Caso 3: existem linhas $i<i'$ de $C$ tais que ambas são não nulas e, sendo $a_{ij}, a_{i'j'}$ os pivôs dessas linhas, $j>j'$.
    
    Ora, como $S$ está na forma escalonada, isso só é possível se $j=n+1$.
    Logo, a $i$-ésima linha de $C$ é nula, o que implica que a $i'$-ésima linha de $S$ é nula, mas $a_{i'j'}$ é pivô de $S$, o que é um absurdo.
\end{itemize}
\end{proof}
\begin{proposition}
    Seja $A \in M_{m \times n}(\mathbb K)$ uma matriz e $B \in M_{m \times n}(\mathbb K)$ uma matriz na forma escalonada equivalente a $B$.
    Seja $S$ o número de pivôs de $B$.

    \begin{itemize}
        \item $m$ representa o número de equações do sistema linear homogêneo $AX=0$.
        \item $n$ representa o número de incógnitas do sistema linear homogêneo $AX=0$.
        \item O número $S\leq m$ representa o número de variávies dependentes do sistema linear homogêneo $AX=0$ -- variáveis que são determinadas a partir das demais.
        \item O número $n-S$ representa o número de variáveis independentes do sistema linear homogêneo $AX=0$ -- variáveis que podem ser escolhidas livremente. Ele é chamado de \emph{grau de liberdade} do sistema linear homogêneo $AX=0$.
        \item O sistema linear homogêneo $AX=0$ possui soluções não triviais se, e somente se $n-S>0$. Assim, se $m<n$, então o sistema linear homogêneo $AX=0$ possui soluções não triviais.
    \end{itemize}
\end{proposition}