\section{Sistemas lineares não homogêneos}

Já possuímos todas as ferramentas teóricas para tratar de sistemas não homogêneos.
O último ingrediente se traduz na seguinte afirmação:

\begin{lemma}
Seja $A\in M_{m\times (n+1)}(\mathbb K)$.
Escreva $A=(a_{ij})_{i, j}$.
Os seguintes sistemas lineares são equivalentes:

\begin{align*}
        \begin{cases}
            a_{11}x_1 + a_{12}x_2 + \cdots + a_{1n}x_n = a_{1(n+1)} \\
            a_{21}x_1 + a_{22}x_2 + \cdots + a_{2n}x_n = a_{2(n+1)} \\
            \vdots \\
            a_{m1}x_1 + a_{m2}x_2 + \cdots + a_{mn}x_n = a_{m(n+1)}.
        \end{cases}
\end{align*}

e:

\begin{align*}
        \begin{cases}
            a_{11}x_1 + a_{12}x_2 + \cdots + a_{1n}x_n +a_{1(n+1)}x_{n+1}=0 \\
            a_{21}x_1 + a_{22}x_2 + \cdots + a_{2n}x_n +a_{2(n+1)}x_{n+1}=0 \\
            \vdots \\
            a_{m1}x_1 + a_{m2}x_2 + \cdots + a_{mn}x_n +a_{m(n+1)}x_{n+1}=0 \\
            x_{n+1}=-1.
        \end{cases}
\end{align*}
\end{lemma}

Isso nos dá um método prático para resolver sistemas lineares não homogêneos.

\begin{itemize}
    \item Dado um sistema linear não homogêneo $AX=b$, podemos formar a matriz aumentada $(A|b)$, em que é adicionada à $A$ uma coluna extra dada pelos elementos de $b$.
    \item O sistema $AX=b$ é, então, equivalente ao sistema homogêneo associado à matriz aumentada $(A|b)X=0$ juntado com a equação extra $x_{n+1}=-1$.
    \item Escalonando-se $(A|b)$ encontramos uma matriz da forma $(S|c)$, onde $S$ está na forma escalonada.
    \item O sistema original é, então, equivalente ao sistema $(S|c)X=c$ junto com a equação extra $x_{n+1}=-1$, que é equivalente ao sistema $SX=c$.
    \item Como acontece com os sistemas homogêneos, as soluções de $SX=c$ são notáveis. Porém, há um caso adicional: se temos uma linha correspondente a uma equação do tipo $0=1$, o sistema não possui solução. Nesse caso, dizemos que o sistema é \emph{impossível}, ou \emph{inconsistente}.
    \item Caso contrário, dizemos que o sistema é \emph{possível}, ou \emph{consistente}, e, como acontece no caso homogêneo, ou teremos infinitas soluções (no caso de alguma coluna de $S$ não possuir pivô, caso no qual dizemos que o sistema é \emph{indeterminado}), ou uma solução única (no caso de todas as colunas de $S$ possuírem pivô, e nesse caso, dizemos que o sistema é \emph{determinado}).
\end{itemize}