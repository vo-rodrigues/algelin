\section{A representação matricial de um sistema linear}

No Brasil, é usual um estudo introdutório sobre sistemas lineares de números reais.

Um sistema linear de números reais é, intuitivamente, uma coleção finita de equações da forma:
\begin{equation}\label{eq:sistemaLinear}\index{sistema linear}
    \begin{cases}
        a_{11}x_1 + a_{12}x_2 + \cdots + a_{1n}x_n = b_1 \\
        a_{21}x_1 + a_{22}x_2 + \cdots + a_{2n}x_n = b_2 \\
        \vdots \\
        a_{m1}x_1 + a_{m2}x_2 + \cdots + a_{mn}x_n = b_m
    \end{cases}
\end{equation}

Em que cada $a_{ij}$ e cada $b_i$ é um número real, para $1 \leq i \leq m$ e $1 \leq j \leq n$.
Os números $a_{ij}$ são chamados de \emph{coeficientes} do sistema, e os números $b_i$ são chamados de \emph{termos independentes} do sistema.

Caso todos os $b_i$ sejam iguais a zero, dizemos que o sistema é dito \emph{homogêneo}.

Os símbolos $x_1, x_2, \ldots, x_n$ são chamados de \emph{variáveis} do sistema.
Resolver um sistema linear consiste em determinar (todos) os valores reais para as variáveis $x_1, x_2, \ldots, x_n$ que satisfaçam todas as equações do sistema.

\begin{example}
    Considere o seguinte sistema linear:
    \begin{equation}\label{eq:exemploVar1}
        \begin{cases}
            2x_1 + 3x_2 = 5 \\
            x_1 - x_2 = 1
        \end{cases}
    \end{equation}

    Nesse caso, temos $m=2$, $n=2$, $a_{11}=2$, $a_{12}=3$, $a_{21}=1$, $a_{22}=-1$, $b_1=5$ e $b_2=1$.
    As variáveis do sistema são $x_1$ e $x_2$.
    O leitor pode verificar, por meios elementares, que a única solução do sistema é dada por $(x_1, x_2)=(2, 1)$.
\end{example}

Enfatizamos que não é um requisito para a leitura deste texto o conhecimento prévio de técnicas para resolver sistemas lineares, uma vez que um dos objetivos deste capítulo é apresentar técnicas gerais que resolvam qualquer sistema linear.

Aproveitando o exemplo anterior, considere o seguinte sistema linear:

    \begin{equation}\label{eq:exemploVar2}
        \begin{cases}
            2y_1 + 3y_2 = 5 \\
            y_1 - y_2 = 1
        \end{cases}
    \end{equation}

O sistema linear da Equação~\eqref{eq:exemploVar2} é essencialmente o mesmo que o sistema linear da Equação~\eqref{eq:exemploVar1}, apenas com as variáveis renomeadas.
Por óbvio, as soluções de ambos os sistemas devem ser as mesmas.
Por isso, ao estudar soluções de sistemas lineares, nos preocupamos apenas com os coeficientes dos mesmos.

Assim, para estudar sistemas lineares, podemos definir formalmente objetos matemáticos que agem como ``tabelas'' de números.
Tais objetos são as \emph{matrizes}.

\begin{definition}\index{matriz}
    Seja $A$ um anel e $m, n$ inteiros positivos.
    Uma \emph{matriz de ordem $m \times n$} (de $m$ linhas e $n$ colunas) com entradas em $A$ é uma família $A=(a_{ij})_{1 \leq i \leq m, 1 \leq j \leq n}$ de elementos de $A$.

    Podemos ainda a denotar por:

    \begin{equation*}
        A = \begin{pmatrix}
            a_{11} & a_{12} & \cdots & a_{1n} \\
            a_{21} & a_{22} & \cdots & a_{2n} \\
            \vdots & \vdots & \ddots & \vdots \\
            a_{m1} & a_{m2} & \cdots & a_{mn}
        \end{pmatrix}
    \end{equation*}

    Ou ainda, por:
    \begin{equation*}
        A = \begin{pmatrix}
            a_{11} & a_{12} & \cdots & a_{1n} \\
            a_{21} & a_{22} & \cdots & a_{2n} \\
            \vdots & \vdots & \ddots & \vdots \\
            a_{m1} & a_{m2} & \cdots & a_{mn}
        \end{pmatrix}
    \end{equation*}

    E, finalmente, quando $m, n$ são subentendidos, podemos escrever $A=(a_{ij})_{i, j}$.

    O conjunto de todas as matrizes de ordem $m \times n$ com entradas em $A$ é denotado por $M_{m \times n}(A)$.
    Quando $m=n$, dizemos que a matriz é \emph{quadrada} de ordem $n$ e denotamos por $M_n(A)$.

    Os elementos $a_{ij}$ são chamados de \emph{coeficientes} de $A$.
\end{definition}

Neste texto, daremos particular atenção para o corpo $A=\mathbb R$, ou, com maior generalidade, para corpos.

\begin{example}
    Seja $(a_{ij})_{i, j}\in M_{2}(\mathbb R)$ a matriz dada por:
    \begin{equation*}
        A = \begin{pmatrix}
            2 & 3 \\
            1 & -1
        \end{pmatrix}
    \end{equation*}
    Então seus coeficientes são $a_{11}=2$, $a_{12}=3$, $a_{21}=1$ e $a_{22}=-1$.
\end{example}

Neste texto, estudaremos sistemas lineares sobre corpos, com ênfase especial no corpo dos números reais $\mathbb R$.
O estudo de sistemas lineares sobre corpos, conforme veremos, tem relação direta com o estudo do que é conhecido por \emph{espaço vetorial}, e este é objeto da \emph{Álgebra Linear}.

O estudo de sistemas lineares com coeficientes em anéis é mais geral, e, apesar de possuir muitas semelhanças com o que será apresentado, possui também muitas características próprias estudadas na Teoria de Módulos.
Tal estudo não é objeto deste texto.

\begin{definition}\index{sistema linear!matricial}
    Seja $\mathbb K$ um corpo e $A=(a_{ij})_{i, j}\in M_{m \times n}(\mathbb K)$ uma matriz.
    O \emph{sistema linear homogêneo associado a $A$} é o sistema linear dado por:
    \begin{equation}\label{eq:sistemaMatricial}
        \begin{cases}
            a_{11}x_1 + a_{12}x_2 + \cdots + a_{1n}x_n = 0 \\
            a_{21}x_1 + a_{22}x_2 + \cdots + a_{2n}x_n = 0 \\
            \vdots \\
            a_{m1}x_1 + a_{m2}x_2 + \cdots + a_{mn}x_n = 0
        \end{cases}
    \end{equation}

    Tal sistema linear é denotado por $AX=0$.
    O \emph{conjunto solução do sistema linear homogêneo de $A$} é o conjunto de todas as $n$-uplas $(x_1, x_2, \ldots, x_n)\in \mathbb K^n$ que satisfazem o sistema linear da Equação~\eqref{eq:sistemaMatricial}.

    Se, adicionalmente, $b=(b_1, b_2, \ldots, b_m)\in \mathbb K^m$, o \emph{sistema linear associado a $A$ e $b$} é o sistema linear dado por:
    \begin{equation}\label{eq:sistemaMatricial2}
        \begin{cases}
            a_{11}x_1 + a_{12}x_2 + \cdots + a_{1n}x_n = b_1 \\
            a_{21}x_1 + a_{22}x_2 + \cdots + a_{2n}x_n = b_2 \\
            \vdots \\
            a_{m1}x_1 + a_{m2}x_2 + \cdots + a_{mn}x_n = b_m
        \end{cases}
    \end{equation}

    Tal sistema linear é denotado por $AX=b$.

    O \emph{conjunto solução do sistema linear de $A$ com coeficientes independentes $b$} é o conjunto de todas as $n$-uplas $(x_1, x_2, \ldots, x_n)\in \mathbb K^n$ que satisfazem o sistema linear da Equação~\eqref{eq:sistemaMatricial2}.
\end{definition}

É imediato que todo sistema linear pode ser expresso na forma matricial, como acima.
Assim, o estudo de matrizes pode ser uma ferramenta poderosa no estudo de sistemas lineares.
\begin{example}
    Seja $(a_{ij})_{i, j}\in M_{2}(\mathbb R)$ a matriz dada por:
    \begin{equation*}
        A = \begin{pmatrix}
            2 & 3 \\
            1 & -1
        \end{pmatrix}
    \end{equation*}

    O sistema linear homogêneo associado a $A$ é o sistema linear dado por:
    \begin{equation*}
        \begin{cases}
            2x_1 + 3x_2 = 0 \\
            x_1 - x_2 = 0
        \end{cases}
    \end{equation*}
\end{example}
Conforme vimos, o produto matricial é uma operação que nasce naturalmente do estudo de sistemas lineares.

Outra operação pertinente é a soma de matrizes, motivada pelo seguinte fato:

Se $A, B \in M_{m \times n}(\mathbb K)$ são matrizes e $x\in \mathbb K^n$ é solução dos sistemas lineares homogêneos $AX=0$ e $BX=0$, então, somando as equações linha-a-linha, $x$ será solução do sistema linear resultante.
Os coeficientes desse sistema são dados pela soma dos coeficientes de $A$ e $B$.

\begin{definition}
    Sejam $A=(a_{ij})_{i, j}\in M_{m \times n}(\mathbb K)$ e $B=(b_{ij})_{i, j}\in M_{m \times n}(\mathbb K)$ matrizes.
    A \emph{soma de matrizes} $A+B$ é a matriz dada como a seguir.

    \begin{equation*}
        A+B = (a_{ij}+b_{ij})_{i, j}\in M_{m \times n}(\mathbb K)
    \end{equation*}

    Ou seja:
    
    \begin{equation*}
    \begin{pmatrix}
        a_{11}& \cdots & a_{1n} \\
        \vdots & \ddots & \vdots \\
        a_{m1} & \cdots & a_{mn}
    \end{pmatrix}
    +
    \begin{pmatrix}
        b_{11}& \cdots & b_{1n} \\
        \vdots & \ddots & \vdots \\
        b_{m1} & \cdots & b_{mn}
    \end{pmatrix}
    =
    \begin{pmatrix}
        a_{11}+b_{11}& \cdots & a_{1n}+b_{1n} \\
        \vdots & \ddots & \vdots \\
        a_{m1}+b_{m1} & \cdots & a_{mn}+b_{mn}
    \end{pmatrix}
    \end{equation*}
\end{definition}