\section{Autovalores, autovetores e autoespaços}
Note que, se o que foi descrito no parágrafo anterior for possível, cada elemento $v$ da base $\mathcal{B}$ deve satisfazer $T(v) = \lambda v$ para algum escalar $\lambda \in \mathbb{K}$ e ser não nulo.

\begin{definition}
Seja $V$ um espaço vetorial sobre um corpo $\mathbb{K}$ e $T: V \to V$ um operador linear.
Um vetor $v \in V \setminus \{0\}$ é chamado de \emph{autovetor} de $T$ se existir $\lambda \in \mathbb{K}$ tal que $T(v) = \lambda v$.

O escalar $\lambda$ é chamado de \emph{autovalor} de $T$ associado ao autovetor $v$.

O \emph{autoespaço} de $T$ associado a um autovalor $\lambda$ é o conjunto
\begin{equation*}
    \aut_\lambda(T) = \{v \in V: T(v) = \lambda v\}.
\end{equation*}
Assim, $\aut_\lambda(T)$ é o conjunto de todos os autovetores de $T$ associados a $\lambda$, e o vetor nulo.

O \emph{espectro} de $T$ é o conjunto de todos os autovalores de $T$.
A \emph{dimensão geométrica} de um autovalor $\lambda$ de $T$ é a dimensão do autoespaço $\aut_\lambda(T)$.
\end{definition}

\begin{definition}
    Seja $V$ um espaço vetorial sobre um corpo $\mathbb{K}$ e $T: V \to V$ um operador linear.
    
    Dizemos que $T$ é \emph{diagonalizável} se existir uma base de $V$ formada por autovetores de $T$.
\end{definition}

\begin{proposition}
    Seja $V$ um espaço vetorial sobre um corpo $\mathbb{K}$ de dimensão $n\geq 1$ e $T: V \to V$ um operador linear.

    Então $n$ é diagonalizável se, e somente se, existe uma base ordenada $\mathcal{B}$ de $V$ tal que a matriz associada $[T]_{\mathcal{B}}$ seja uma matriz diagonal.
\end{proposition}

\begin{proof}
    Suponha que $T$ seja diagonalizável.
    Existe $\mathcal B$ uma base ordenada composta de autovetores de $V$. Para cada $v$ na base, $T(v)= \lambda v$ para algum escalar $\lambda \in \mathbb{K}$, logo, a matriz associada $[T]_{\mathcal{B}}$ é diagonal.

    Reciprocamente, se $[T]_{\mathcal{B}}$ é uma matriz diagonal para alguma base ordenada $\mathcal B$ de $V$, então, para cada vetor $v$ em $\mathcal B$, $T(v) = \lambda v$ para algum escalar $\lambda \in \mathbb{K}$.
    Assim, $v$ é um autovetor de $T$.
\end{proof}

\begin{lemma}
    Na notação acima, $\aut_\lambda(T)$ é um subespaço vetorial de $V$ e $\aut_\lambda(T) = \ker(\lambda \id-T)$.
\end{lemma}
\begin{proof}
    Basta provar a igualdade de conjuntos.
    Para tanto, note que, para todo $v \in V$:

    \begin{equation*}
        v \in \aut_\lambda(T) \iff T(v) = \lambda v \iff \lambda v - T(v) = 0 \iff (\lambda \id - T)(v) = 0 \iff v \in \ker(\lambda \id - T).
    \end{equation*}
\end{proof}

Os autoespaços estão em soma direta.
Isso é:

\begin{proposition}
    Sejam $V$ um espaço vetorial sobre um corpo $\mathbb{K}$, $T: V \to V$ um operador linear, e $\lambda_1, \ldots, \lambda_k \in \mathbb{K}$ autovalores distintos de $T$.
    Então se $v_1 \in \aut_{\lambda_1}(T), \ldots, v_k \in \aut_{\lambda_k}(T)$ são tais que $v_1 + \cdots + v_k = 0$, segue que $v_1 = \cdots = v_k = 0$.
\end{proposition}
\begin{proof}
    Faremos indução em $k$.

    Caso base: $k=1$.
    Nesse caso, se $v_1 \in \aut_{\lambda_1}(T)$ é tal que $v_1 = 0$, então $v_1 = 0$.

    Passo indutivo: suponha que a proposição seja verdadeira para algum $k \geq 1$.
    Seja $v_1 \in \aut_{\lambda_1}(T), \ldots, v_{k+1} \in \aut_{\lambda_{k+1}}(T)$ tais que
    \begin{equation*}
        v_1 +  \cdots + v_k+ v_{k+1} = 0.
    \end{equation*}
    Aplicando $T$ em ambos os lados da equação acima, temos:
    \begin{equation*}
        T(v_1) +  \cdots + T(v_k) + T(v_{k+1}) = 0.
    \end{equation*}
    Ou seja,
    \begin{equation*}
        \lambda_1 v_1 + \cdots + \lambda_k v_k + \lambda_{k+1} v_{k+1} = 0.
    \end{equation*}
    Multiplicando a primeira equação por $\lambda_{k+1}$ e subtraindo da segunda, obtemos:
    \begin{equation*}
        (\lambda_1 - \lambda_{k+1}) v_1 + (\lambda_2 - \lambda_{k+1}) v_2 + \cdots + (\lambda_k - \lambda_{k+1}) v_k = 0.
    \end{equation*}
    Pelo passo indutivo, segue que, para cada $i \in \{1, \ldots, k\}$, temos que $(\lambda_i - \lambda_{k+1}) v_i = 0$.
    Como os autovalores são distintos, segue que $v_i = 0$ para cada $i \in \{1, \ldots, k\}$.
    Voltando à equação inicial, segue que $v_{k+1} = 0$.
\end{proof}
\begin{corollary}
    Sejam $V$ um espaço vetorial sobre um corpo $\mathbb{K}$, $T: V \to V$ um operador linear, e $\lambda_1, \ldots, \lambda_k \in \mathbb{K}$ autovalores distintos de $T$.
    Então, se para cada $i \in \{1, \ldots, k\}$, $\mathcal{B}_i$ é um conjunto LI de $\aut_{\lambda_i}(T)$, a união $\mathcal{B}_1 \cup \mathcal{B}_2 \cup \cdots \cup \mathcal{B}_k$ é disjunta, e é LI em $V$.
\end{corollary}
\begin{proof}
    Para cada $i \in \{1, \ldots, k\}$, seja $\mathcal B_i=\{v_{i1}, v_{i2}, \ldots, v_{i n_i}\}$ uma base de $\aut_{\lambda_i}(T)$, escrita sem repetições.
    Seja $c_{ij} \in \mathbb{K}$ tais que
    \begin{equation*}
        \sum_{i=1}^k \sum_{j=1}^{n_i} c_{ij} v_{ij} = 0.
    \end{equation*}
    Pelo lema, segue que, para cada $i \in \{1, \ldots, k\}$, temos que
    \begin{equation*}
        \sum_{j=1}^{n_i} c_{ij} v_{ij} = 0.
    \end{equation*}
    Como $\mathcal B_i$ é LI em $\aut_{\lambda_i}(T)$, segue que, para cada $i \in \{1, \ldots, k\}$ e $j \in \{1, \ldots, n_i\}$, temos que $c_{ij} = 0$.

    Para ver que a união é disjunta, note que, se existisse $v\mathcal B_i\cap \mathcal B_{i'}$, com $i \neq i'$, então teríamos $T(v)=\lambda_i v$ e $T(v) = \lambda_{i'} v$, o que implicaria que $(\lambda_i - \lambda_{i'}) v = 0$.
    Como os autovalores são distintos, segue que $v = 0$, o que é um absurdo, pois conjuntos LI não contêm o vetor nulo.
\end{proof}
\begin{corollary}
    Seja $V$ um espaço vetorial sobre um corpo $\mathbb{K}$ de dimensão finita positiva $n$ e $T: V \to V$ um operador linear.

    Seja $S$ o espectro de $T$.

    Então:

    \begin{enumerate}[label=(\roman*)]
        \item O número de autovalores distintos de $T$ é, no máximo, $n$.
        \item A soma das dimensões geométricas dos autovalores de $T$ é, no máximo, $n$.
        \item $T$ é diagonalizável se, e somente se, a soma das dimensões geométricas dos autovalores de $T$ é igual a $n$.
        Em símbolos, $T$ é diagonalizável se, e somente se,
        \begin{equation*}
            \sum_{\lambda \in S} \dim(\aut_\lambda(T)) = n.
        \end{equation*}

        Além disso, nesse caso, toda base $\mathcal B$ de $V$ se decompõe como união disjunta de bases dos autoespaços de $T$.
        \item Se $T$ possui $n$ autovalores distintos, então $T$ é diagonalizável.
    \end{enumerate}
\end{corollary}
\begin{proof}
    \begin{enumerate}[label=(\roman*)]
        \item Se $T$ tivesse mais que $n$ autovalores distintos, então, escolhendo um autovetor não nulo para cada autovalor, teríamos mais que $n$ vetores LI em $V$, o que é um absurdo.
        \item Similar à (i): se a soma das dimensões geométricas dos autovalores de $T$ fosse maior que $n$, então, escolhendo uma base para cada autoespaço e unindo tais bases, obteríamos mais que $n$ vetores LI em $V$, o que é um absurdo.
        \item ($\Rightarrow$) Se $T$ é diagonalizável, então existe uma base $\mathcal{B}$ de $V$ formada por autovetores de $T$.
        
        Para cada $\lambda \in S$, seja $\mathcal{B}_\lambda = \mathcal{B} \cap \aut_\lambda(T)$ e $n_\lambda=\dim(\aut_\lambda(T))$.

        Como todo elemento de $\mathcal{B}$ é um autovetor de $T$, segue que $\mathcal B=\dot \bigcup_{\lambda \in S} \mathcal{B}_\lambda$.
        Cada $\mathcal B_\lambda$ é LI em $\aut_\lambda(T)$.
        Logo, $|\mathcal B_\lambda| \leq n_\lambda$.
        
        Além disso, $n=|\mathcal B|=\sum_{\lambda \in S} |\mathcal B_\lambda| \leq \sum_{\lambda \in S} n_\lambda$.
        Caso, para algum $\lambda \in S$, tivéssemos $|\mathcal B_\lambda| < n_\lambda$, então teríamos que a desigualdade intermediária é estrita o que nos dá um absurdo.
        Logo, para cada $\lambda \in S$, temos que $|\mathcal B_\lambda| = n_\lambda$, o que implica que $\mathcal B_\lambda$ é uma base de $\aut_\lambda(T)$ e que $n = \sum_{\lambda \in S} n_\lambda$.

        ($\Leftarrow$) Suponha que $\sum_{\lambda \in S} \dim(\aut_\lambda(T)) = n$.
        Para cada $\lambda \in S$, seja $\mathcal{B}_\lambda$ uma base de $\aut_\lambda(T)$.
        Então, pela proposição anterior, a união disjunta $\mathcal{B} = \dot \bigcup_{\lambda \in S} \mathcal{B}_\lambda$ é LI em $V$.
        Como o número de elementos de $\mathcal B$ é igual a $n=\dim V$, segue que $\mathcal B$ é uma base de $V$ formada por autovetores de $T$.
        Assim, $T$ é diagonalizável.
        \item Decorre dos itens anteriores: se $T$ possui $n$ autovalores distintos, então a soma das dimensões geométricas dos autovalores de $T$ é, no mínimo, $n$, e, no máximo, $n$.
        Assim, a soma das dimensões geométricas dos autovalores de $T$ é igual a $n$.
        Pelo item (iii), segue que $T$ é diagonalizável.
    \end{enumerate}
\end{proof}
Como encontrar autovalores e autovetores?
Uma forma natural é buscar autovetores em coordenadas.
Para tanto, temos o seguinte.

\begin{proposition}
    Seja $V$ um espaço vetorial sobre um corpo $\mathbb{K}$ de dimensão finita positiva, $T: V \to V$ um operador linear, e $\mathcal{B}$ uma base ordenada de $V$.
    
    Então $\lambda \in \mathbb{K}$ é um autovalor de $T$ se, e somente se, $\det(\lambda I_n - [T]_{\mathcal{B}}) = 0$.
\end{proposition}
\begin{proof}
    Note que $\lambda$ é um autovalor de $T$ se, e somente se, existe $v \in V \setminus \{0\}$ tal que $T(v) = \lambda v$.
    Isso é equivalente a dizer que existe $v \in \ker(\lambda \id - T) \setminus \{0\}$, ou seja, tal que $(\lambda \id - T)(v) = 0$.

    Em coordenadas, isso é equivalente a dizer que existem $\alpha_1, \ldots, \alpha_n \in \mathbb{K}$, não todos nulos, tais que
    \begin{equation*}
        (\lambda I_n - [T]_{\mathcal{B}}) \begin{pmatrix}
        \alpha_1\\
        \alpha_2\\
        \vdots\\
        \alpha_n
        \end{pmatrix} = 0.
    \end{equation*}
    Ou seja, que o sistema homogêneo associado à matriz $\lambda I_n - [T]_{\mathcal{B}}$ tem solução não trivial.

    Pelo visto no capítulo anterior, isso é equivalente a dizer que $\det(\lambda I_n - [T]_{\mathcal{B}}) = 0$.
\end{proof}

Os polinômios sobre um corpo formam um anel comutativo.
Logo, podemos considerar as funções determinantes associadas a matrizes cujos coeficientes são polinômios, elementos de $\mathbb K[t]$.
Na notação da proposição anterior, está claro, portanto, que $\lambda$ é um autovalor de $T$ se, e somente se, $\lambda$ é raiz do polinômio $\det(t I_n - [T]_{\mathcal{B}}) \in \mathbb K[t]$.

A proposição abaixo nos mostra que o polinômio $\det(t I_n - [T]_{\mathcal{B}})$ não depende da base escolhida.

\begin{lemma}
    Seja $V$ um espaço vetorial sobre um corpo $\mathbb{K}$ de dimensão finita positiva, $T: V \to V$ um operador linear, e $\mathcal{B}$ e $\mathcal{C}$ duas bases ordenadas de $V$.
    
    Então $\det(t I_n - [T]_{\mathcal{B}}) = \det(t I_n - [T]_{\mathcal{C}})$.
\end{lemma}
\begin{proof}
    Seja $P = [\id]_{\mathcal{C}}^{\mathcal{B}}$ a matriz de mudança de base de $\mathcal{B}$ para $\mathcal{C}$.
    Então, como visto no capítulo anterior, temos que
    \begin{equation*}
        [T]_{\mathcal{C}} = P [T]_{\mathcal{B}} P^{-1}.
    \end{equation*}

    Assim,
    \begin{equation*}
        \begin{aligned}
            \det(t I_n - [T]_{\mathcal{C}}) &= \det(t I_n - P [T]_{\mathcal{B}} P^{-1})\\
            &= \det(P (t I_n - [T]_{\mathcal{B}}) P^{-1})\\
            &= \det(P) \det(t I_n - [T]_{\mathcal{B}}) \det(P^{-1})\\
            &= \det(t I_n - [T]_{\mathcal{B}}).
        \end{aligned}
    \end{equation*}
\end{proof}

Isso nos permite definir o polinômio característico de um operador linear.