\section{Autovalores, autovetores e autoespaços}
Note que, se o que foi descrito no parágrafo anterior for possível, cada elemento $v$ da base $\mathcal{B}$ deve satisfazer $T(v) = \lambda v$ para algum escalar $\lambda \in \mathbb{K}$ e ser não nulo.

\begin{definition}
Seja $V$ um espaço vetorial sobre um corpo $\mathbb{K}$ e $T: V \to V$ um operador linear.
Um vetor $v \in V \setminus \{0\}$ é chamado de \emph{autovetor} de $T$ se existir $\lambda \in \mathbb{K}$ tal que $T(v) = \lambda v$.

O escalar $\lambda$ é chamado de \emph{autovalor} de $T$ associado ao autovetor $v$.

O \emph{autoespaço} de $T$ associado a um autovalor $\lambda$ é o conjunto
\begin{equation*}
    \aut_\lambda(T) = \{v \in V: T(v) = \lambda v\}.
\end{equation*}
Assim, $\aut_\lambda(T)$ é o conjunto de todos os autovetores de $T$ associados a $\lambda$, e o vetor nulo.
\end{definition}

\begin{lemma}
    Na notação acima, $\aut_\lambda(T)$ é um subespaço vetorial de $V$ e $\aut_\lambda(T) = \ker(\lambda \id-T)$.
\end{lemma}
\begin{proof}
    Basta provar a igualdade de conjuntos.
    Para tanto, note que, para todo $v \in V$:

    \begin{equation*}
        v \in \aut_\lambda(T) \iff T(v) = \lambda v \iff \lambda v - T(v) = 0 \iff (\lambda \id - T)(v) = 0 \iff v \in \ker(\lambda \id - T).
    \end{equation*}
\end{proof}

Como encontrar autovalores e autovetores?
Uma forma natural é buscar autovetores em coordenadas.
Para tanto, temos o seguinte.

\begin{proposition}
    Seja $V$ um espaço vetorial sobre um corpo $\mathbb{K}$ de dimensão finita positiva, $T: V \to V$ um operador linear, e $\mathcal{B}$ uma base ordenada de $V$.
    
    Então $\lambda \in \mathbb{K}$ é um autovalor de $T$ se, e somente se, $\det(\lambda I_n - [T]_{\mathcal{B}}) = 0$.
\end{proposition}
\begin{proof}
    Note que $\lambda$ é um autovalor de $T$ se, e somente se, existe $v \in V \setminus \{0\}$ tal que $T(v) = \lambda v$.
    Isso é equivalente a dizer que existe $v \in \ker(\lambda \id - T) \setminus \{0\}$, ou seja, tal que $(\lambda \id - T)(v) = 0$.

    Em coordenadas, isso é equivalente a dizer que existem $\alpha_1, \ldots, \alpha_n \in \mathbb{K}$, não todos nulos, tais que
    \begin{equation*}
        (\lambda I_n - [T]_{\mathcal{B}}) \begin{pmatrix}
        \alpha_1\\
        \alpha_2\\
        \vdots\\
        \alpha_n
        \end{pmatrix} = 0.
    \end{equation*}
    Ou seja, que o sistema homogêneo associado à matriz $\lambda I_n - [T]_{\mathcal{B}}$ tem solução não trivial.

    Pelo visto no capítulo anterior, isso é equivalente a dizer que $\det(\lambda I_n - [T]_{\mathcal{B}}) = 0$.
\end{proof}

Os polinômios sobre um corpo formam um anel comutativo.
Logo, podemos considerar as funções determinantes associadas a matrizes cujos coeficientes são polinômios, elementos de $\mathbb K[t]$.
Na notação da proposição anterior, está claro, portanto, que $\lambda$ é um autovalor de $T$ se, e somente se, $\lambda$ é raiz do polinômio $\det(t I_n - [T]_{\mathcal{B}}) \in \mathbb K[t]$.

A proposição abaixo nos mostra que o polinômio $\det(t I_n - [T]_{\mathcal{B}})$ não depende da base escolhida.

\begin{lemma}
    Seja $V$ um espaço vetorial sobre um corpo $\mathbb{K}$ de dimensão finita positiva, $T: V \to V$ um operador linear, e $\mathcal{B}$ e $\mathcal{C}$ duas bases ordenadas de $V$.
    
    Então $\det(t I_n - [T]_{\mathcal{B}}) = \det(t I_n - [T]_{\mathcal{C}})$.
\end{lemma}
\begin{proof}
    Seja $P = [\id]_{\mathcal{C}}^{\mathcal{B}}$ a matriz de mudança de base de $\mathcal{B}$ para $\mathcal{C}$.
    Então, como visto no capítulo anterior, temos que
    \begin{equation*}
        [T]_{\mathcal{C}} = P [T]_{\mathcal{B}} P^{-1}.
    \end{equation*}

    Assim,
    \begin{equation*}
        \begin{aligned}
            \det(t I_n - [T]_{\mathcal{C}}) &= \det(t I_n - P [T]_{\mathcal{B}} P^{-1})\\
            &= \det(P (t I_n - [T]_{\mathcal{B}}) P^{-1})\\
            &= \det(P) \det(t I_n - [T]_{\mathcal{B}}) \det(P^{-1})\\
            &= \det(t I_n - [T]_{\mathcal{B}}).
        \end{aligned}
    \end{equation*}
\end{proof}

Isso nos permite definir o polinômio característico de um operador linear.