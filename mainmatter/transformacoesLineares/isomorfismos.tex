\section{Isomorfismos}
Conforme já mencionado, um isomorfismo é uma transformação linear bijetora.
Se há um isomorfismo entre dois espaços vetoriais, dizemos que eles são isomorfos.
Nesse caso, no que diz respeito à estrutura de espaço vetorial, eles são essencialmente o mesmo espaço, e a função isomorfismo pode ser vista como um dispositivo de ``tradução'' entre os dois espaços.

A relação de isomorfismo entre espaços vetoriais é uma relação de equivalência na classe dos espaços vetoriais sobre um mesmo corpo.

\begin{proposition}
    Sejam $U$, $V$ e $W$ espaços vetoriais sobre um corpo $\mathbb K$.
    Então:
    \begin{enumerate}[label=(\alph*)]
        \item $\id_U: U \to U$ é um isomorfismo.
        Portanto, $U$ é isomorfo a si mesmo.
        \item Se $T: U \to V$ é um isomorfismo e $S: V \to W$ é um isomorfismo, então $S \circ T: U \to W$ é um isomorfismo.
        Portanto, se $U$ é isomorfo a $V$ e $V$ é isomorfo a $W$, então $U$ é isomorfo a $W$.
        \item Se $T: U \to V$ é um isomorfismo, então sua inversa $T^{-1}: V \to U$ é um isomorfismo.
        Portanto, se $U$ é isomorfo a $V$, então $V$ é isomorfo a $U$.
    \end{enumerate}
\end{proposition}
\begin{proof}
    (a) É claro que $\id_U$ é bijetora é linear.
    Para (b), é um fato geral que a composição de funções bijetoras é bijetora, e, como já vimos, a composição de funções lineares é linear.

    Para (c), se $T$ é um isomorfismo, então sua inversa $T^{-1}$ também é bijetora.
    Resta ver que $T^{-1}$ é linear.

    Sejam $u, v \in V$ e $\alpha \in \mathbb K$.
    Então existem $x, y \in U$ tais que $T(x) = u$ e $T(y) = v$.
    Assim:
    \begin{equation*}
        \begin{aligned}
            T^{-1}(u + \alpha v) &= T^{-1}(T(x) + \alpha T(y))\\
            &= T^{-1}(T(x + \alpha y))\\
            &= x + \alpha y\\
            &= T^{-1}(u) + \alpha T^{-1}(v).
        \end{aligned}
    \end{equation*}
\end{proof}

Abaixo, daremos um importante exemplo de isomorfismo.
\begin{definition}
    Seja $V$ um espaço vetorial sobre um corpo $\mathbb K$, com $1\leq \dim V=n<\infty$.
    Para cada $x \in V$, o vetor coluna das coordenadas de $x$ na base ordenada $\mathcal B$, denotado por $[x]_{\mathcal B}$, é dado por:
    \begin{equation*}
        [x]_{\mathcal B} =
        \begin{pmatrix}
            \alpha_1\\
            \alpha_2\\
            \vdots\\
            \alpha_n
        \end{pmatrix},
    \end{equation*}
    em que $(\alpha_1, \alpha_2, \dots, \alpha_n) = (x)_{\mathcal B}$, ou  seja, em que $x = \sum_{i=1}^n \alpha_i v_i$ e $\mathcal B = (v_1, v_2, \dots, v_n)$.
\end{definition}

Da mesma forma que a transformação $x \mapsto (x)_{\mathcal B}$ é um isomorfismo entre $V$ e $\mathbb K^n$, a transformação $x \mapsto [x]_{\mathcal B}$ também é um isomorfismo entre $V$ e $M_{n\times 1}(\mathbb K)$.

\begin{lemma}
    Sejam $V$ um espaço vetorial sobre um corpo $\mathbb K$, com $1\leq \dim V=n<\infty$, e $\mathcal B$ uma base ordenada de $V$.
    A transformação $[\,\cdot\,]_{\mathcal B}: V \to M_{n\times 1}(\mathbb K)$ é um isomorfismo.
\end{lemma}
\begin{proof}
    Verificaremos a linearidade.
    Sejam $u, w \in V$ e $\beta \in \mathbb K$, com $(\alpha_1, \dots, \alpha_n)=(u)_{\mathcal B}$ e $(\gamma_1, \dots, \gamma_n)=(w)_{\mathcal B}$.

    Sejam as coordenadas de $u$ e $w$ na base $\mathcal B$ dadas por $(\alpha_1, \dots, \alpha_n)$ e $(\gamma_1, \dots, \gamma_n)$, respectivamente.
    Então as coordenadas de $u+w$ e de $\beta w$ na base $\mathcal B$ são dadas por $(\alpha_1 + \gamma_1, \dots, \alpha_n + \gamma_n)$ e $(\beta \gamma_1, \dots, \beta \gamma_n)$, respectivamente.
    Logo:
    \begin{equation*}
        \begin{aligned}
            \left[u + \beta w\right]_{\mathcal B} &= 
            \begin{pmatrix}
                \alpha_1 + \beta \gamma_1\\
                \alpha_2 + \beta \gamma_2\\
                \vdots\\
                \alpha_n + \beta \gamma_n
            \end{pmatrix}\\
            &=
            \begin{pmatrix}
                \alpha_1\\
                \alpha_2\\
                \vdots\\
                \alpha_n
            \end{pmatrix}
            +
            \beta
            \begin{pmatrix}
                \gamma_1\\
                \gamma_2\\
                \vdots\\
                \gamma_n
            \end{pmatrix}\\
            &=
            [u]_{\mathcal B} + \beta [w]_{\mathcal B}.
        \end{aligned}
    \end{equation*}
    O que prova que $[\,\cdot\,]_{\mathcal B}$ é linear.

    Para a injetividade, veremos que o núcleo é trivial.
    Seja $u \in V$ tal que $[u]_{\mathcal B} = 0$.
    Então as coordenadas de $u$ na base $\mathcal B$ são todas nulas, ou seja, $u = 0$.

    Para a sobrejetividade, temos, pelo Teorema do Núcleo e Imagem, que $\dim(\operatorname{Im}([\,\cdot\,]_{\mathcal B})) = \dim(V)=n=\dim(M_{n\times 1}(\mathbb K))$.
\end{proof}