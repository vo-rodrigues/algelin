\section{O núcleo e a imagem de uma transformação linear}

\begin{definition}
    Seja $T: V \to W$ uma transformação linear.
    O \emph{núcleo} de $T$ é o conjunto $\ker(T) = \{v \in V : T(v) = 0_W\}$.
    A \emph{imagem} de $T$ é o conjunto $\Img(T) = \{T(v) : v \in V\}$.
\end{definition}
\begin{proposition}
    Seja $T: V \to W$ uma transformação linear.
    Então, $\ker(T)$ é um subespaço de $V$ e $\Img(T)$ é um subespaço de $W$.
\end{proposition}

\begin{lemma}
    Seja $T: V \to W$ uma transformação linear.
    Então, $\ker(T)$ é um subespaço de $V$ e $\Img(T)$ é um subespaço de $W$.
\end{lemma}
\begin{proof}
    Vejamos que $\ker(T)$ é um subespaço de $V$.
    \begin{itemize}
        \item $0_V \in \ker(T)$, pois $T(0_V) = 0_W$.
        \item Se $u, v \in \ker(T)$ e $\alpha \in \mathbb K$, então $T(u) = 0_W$ e $T(v) = 0_W$.
        Temos o seguinte:
        \begin{equation*}
            T(u + \alpha v) = T(u) + \alpha T(v) = 0_W + \alpha 0_W = 0_W.
        \end{equation*}
    \end{itemize}
    Portanto, $\ker(T)$ é um subespaço de $V$.
    Vejamos que $\Img(T)$ é um subespaço de $W$.
    \begin{itemize}
        \item $0_W \in \Img(T)$, pois $T(0_V) = 0_W$.
        \item Se $x, y \in \Img(T)$ e $\alpha \in \mathbb K$, então existem $u, v \in V$ tais que $x = T(u)$ e $y = T(v)$.
        Temos o seguinte:
        \begin{equation*}
            x + \alpha y = T(u) + \alpha T(v) = T(u + \alpha v) \in \Img(T),
        \end{equation*}
    \end{itemize}
\end{proof}

\begin{lemma}
Seja $T: V \to W$ uma transformação linear.
Então $T$ é injetora se, e somente se, para todo conjunto linearmente independente $L \subseteq V$, o conjunto $T[L]=\{T(v): v \in L\}$ é linearmente independente em $W$.
\end{lemma}
\begin{proof}
Seja $T$ injetora.
Seja $L\subseteq V$ um conjunto linearmente independente.
Sejam $\alpha_1, \ldots, \alpha_n \in \mathbb K$ e $v_1, \ldots, v_n \in L$ distintos tais que $\sum_{i=1}^n \alpha_i T(v_i) = 0_W$.
Então $T\left(\sum_{i=1}^n \alpha_i v_i\right) = 0_W$.
Como $T$ é injetora, temos que $\sum_{i=1}^n \alpha_i v_i = 0_V$.
Como $L$ é linearmente independente, temos que $\alpha_i = 0$ para todo $i$.
Assim, $T[L]$ é linearmente independente.

Reciprocamente, se $T$ não é injetora, então existe $v \in V$, $v \neq 0_V$, tal que $T(v) = 0_W$.
Então, o conjunto $L = \{v\}$ é linearmente independente, mas $T[L] = \{0_W\}$ não é linearmente independente.
\end{proof}

\begin{lemma}
Seja $T: V \to W$ uma transformação linear.
Então $T$ é sobrejetora se, e somente se, para todo conjunto gerador $G \subseteq W$, o conjunto $T[G]=\{T(v): v \in G\}$ é um conjunto gerador de $W$.
\end{lemma}
\begin{proof}
Seja $T$ sobrejetora.
Seja $G\subseteq V$ um conjunto gerador de $V$.
Seja $w \in W$.
Como $T$ é sobrejetora, existe $v \in V$ tal que $T(v) = w$.
Como $G$ gera $V$, existem $u_1, \ldots, u_m \in G$ e $\alpha_1, \ldots, \alpha_m \in \mathbb K$ tais que $v = \sum_{i=1}^m \alpha_i u_i$.
Então:
\begin{equation*}
    w = T(v) = T\left(\sum_{i=1}^m \alpha_i u_i\right) = \sum_{i=1}^m \alpha_i T(u_i),
\end{equation*}
logo, $T[G]$ gera $W$.

Reciprocamente, se $T$ não é sobrejetora, $V$ é um conjunto gerador de $V$ tal que $T[V]=\Img(T)$ não gera $W$.
\end{proof}

\begin{theorem}[Teorema do núcleo e da imagem]
Seja $T: V \to W$ uma transformação linear. Então:
\begin{equation*}
\dim(V) = \dim(\ker(T)) + \dim(\Img(T)).
\end{equation*}
\end{theorem}
\begin{proof}
Veremos a prova apenas para dimensão finita.
Seja $\mathcal B$ uma base de $\ker(T)$.
Seja $\mathcal C$ um conjunto disjunto de $\mathcal B$ tal que $\mathcal B \cup \mathcal C$ é uma base de $V$.
Mostraremos que $T[\mathcal C]$ é uma base de $\Img(T)$ e que $T|_{\mathcal C}$ é injetora.

Sejam $\alpha_1, \ldots, \alpha_n \in \mathbb K$ e $v_1, \ldots, v_n \in \mathcal C$ distintos tais que $\sum_{i=1}^n \alpha_i T(v_i) = 0_W$.
Então $T\left(\sum_{i=1}^n \alpha_i v_i\right) = 0_W$.
Como $\mathcal B$ é base de $\ker(T)$, temos que $\sum_{i=1}^n \alpha_i v_i \in \langle \mathcal B \rangle$.
Logo, existem $(\beta_j)_{j<m}$ tais que $\sum_{i=1}^n \alpha_i v_i = \sum_{j<m} \beta_j b_j$.
Como $\mathcal B \cup \mathcal C$ é LI, temos que $\alpha_i = 0$ para todo $i$.

Assim, $T[\mathcal C]$ é LI. Além disso, $T|_{\mathcal C}$ é injetora. pois se $T(v) = T(u)$ para $u, v \in \mathcal C$, então $T(v)-T(u)=T(v - u) = 0_W$, e, pelo argumento anterior, $1=-1=0$, um absurdo.

Resta ver que $T[\mathcal C]$ gera $\Img(T)$.
Seja $w \in \Img(T)$.
Então, existe $v \in V$ tal que $T(v) = w$.
Como $\mathcal B \cup \mathcal C$ é base de $V$, existem $(\alpha_i)_{i<n}$ e $(\beta_j)_{j<m}$ tais que $v = \sum_{i<n} \alpha_i b_i + \sum_{j<m} \beta_j c_j$, com $b_i \in \mathcal B$ e $c_j \in \mathcal C$.
Assim:

\begin{equation*}
    w = T(v) = T\left(\sum_{i<n} \alpha_i b_i + \sum_{j<m} \beta_j c_j\right) = \sum_{i<n} \alpha_i T(b_i) + \sum_{j<m} \beta_j T(c_j).
\end{equation*}
Como $T(b_i) = 0_W$ para todo $i$, temos que $w = \sum_{j<m} \beta_j T(c_j)$, logo, $T[\mathcal C]$ gera $\Img(T)$.
Portanto, temos:

\begin{equation*}
\dim(V) = |\mathcal B| + |\mathcal C| = \dim(\ker(T)) + \dim(\Img(T)).
\end{equation*}
\end{proof}

