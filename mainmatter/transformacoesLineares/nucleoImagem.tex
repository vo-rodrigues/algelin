\section{O núcleo e a imagem de uma transformação linear}

\begin{definition}
    Seja $T: V \to W$ uma transformação linear.
    O \emph{núcleo} de $T$ é o conjunto $\ker(T) = \{v \in V : T(v) = 0_W\}$.
    A \emph{imagem} de $T$ é o conjunto $\Img(T) = \{T(v) : v \in V\}$.
\end{definition}
\begin{proposition}
    Seja $T: V \to W$ uma transformação linear.
    Então, $\ker(T)$ é um subespaço de $V$ e $\Img(T)$ é um subespaço de $W$.
\end{proposition}

\begin{lemma}
    Seja $T: V \to W$ uma transformação linear.
    Então, $\ker(T)$ é um subespaço de $V$ e $\Img(T)$ é um subespaço de $W$.
\end{lemma}
\begin{proof}
    Vejamos que $\ker(T)$ é um subespaço de $V$.
    \begin{itemize}
        \item $0_V \in \ker(T)$, pois $T(0_V) = 0_W$.
        \item Se $u, v \in \ker(T)$ e $\alpha \in \mathbb K$, então $T(u) = 0_W$ e $T(v) = 0_W$.
        Temos o seguinte:
        \begin{equation*}
            T(u + \alpha v) = T(u) + \alpha T(v) = 0_W + \alpha 0_W = 0_W.
        \end{equation*}
    \end{itemize}
    Portanto, $\ker(T)$ é um subespaço de $V$.
    Vejamos que $\Img(T)$ é um subespaço de $W$.
    \begin{itemize}
        \item $0_W \in \Img(T)$, pois $T(0_V) = 0_W$.
        \item Se $x, y \in \Img(T)$ e $\alpha \in \mathbb K$, então existem $u, v \in V$ tais que $x = T(u)$ e $y = T(v)$.
        Temos o seguinte:
        \begin{equation*}
            x + \alpha y = T(u) + \alpha T(v) = T(u + \alpha v) \in \Img(T),
        \end{equation*}
    \end{itemize}
\end{proof}