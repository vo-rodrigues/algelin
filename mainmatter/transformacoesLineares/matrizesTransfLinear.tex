\section{Matrizes de transformações lineares}

Quem são as transformações lineares entre espaços vetoriais finitamente gerados?

Primeiro, estudemos uma propriedade relativa a bases.

\begin{theorem}
    Sejam $V$ e $W$ espaços vetoriais sobre um corpo $\mathbb K$, com $1\leq \dim V=n<\infty$ e $\mathcal B$ uma base de $V$.
    Então, para toda $f: \mathcal B\rightarrow W$ função, existe uma única transformação linear $T: V \to W$ tal que $T(v) = f(v)$ para todo $v \in \mathcal B$.
\end{theorem}
\begin{proof}
    Ordenemos a base $\mathcal B$ como $\mathcal C=(v_1, \dots, v_n)$.

    Seja $u \in V$ e $(\alpha_1, \dots, \alpha_n)=(u)_{\mathcal C}$.
    Definimos:
    \begin{equation*}
        T(u) = \sum_{i=1}^n \alpha_i f(v_i).
    \end{equation*}

    $T$ é linear, pois dados $u, w \in V$ e $\beta \in \mathbb K$, com $(\alpha_1, \dots, \alpha_n)=(u)_{\mathcal C}$ e $(\gamma_1, \dots, \gamma_n)=(w)_{\mathcal C}$, temos:
    \begin{equation*}
        \begin{aligned}
            T(u + \beta w) &= T\left(\sum_{i=1}^n (\alpha_i + \beta \gamma_i)v_i\right)\\
            &= \sum_{i=1}^n (\alpha_i + \beta \gamma_i)f(v_i)\\
            &= \sum_{i=1}^n \alpha_i f(v_i) + \beta \sum_{i=1}^n \gamma_i f(v_i)\\
            &= T(u) + \beta T(w).
        \end{aligned}
    \end{equation*}

    Temos que $T$ estende $f$, pois as coordenadas de $v_j$ na base $\mathcal C$ são $(0, \dots, 0, 1, 0, \dots, 0)$ (com $1$ na $j$-ésima posição), logo:
    \begin{equation*}
        T(v_j) = \sum_{i=1}^n \delta_{ij} f(v_i) = f(v_j).
    \end{equation*}
    Finalmente, seja $S: V \to W$ outra transformação linear que estende $f$.
    Seja $u \in V$ e $(\alpha_1, \dots, \alpha_n)=(u)_{\mathcal C}$.
    Então:
    \begin{equation*}
        S(u) = S\left(\sum_{i=1}^n \alpha_i v_i\right) = \sum_{i=1}^n \alpha_i S(v_i) = \sum_{i=1}^n \alpha_i f(v_i) = T(u).
    \end{equation*}
    Assim, $S = T$.
\end{proof}

Quem são todas as transformações lineares de $\mathbb K^n$ em $\mathbb K$?

\begin{example}
    Seja $T: \mathbb K^n \to \mathbb K$ uma transformação linear.
    Temos que $T(x_1, \dots, x_n)= \sum_{i=1}^n x_i T(e_i)$, onde $e_i$ é o $i$-ésimo vetor da base canônica de $\mathbb K^n$.

    Logo, para qualquer escolha $T(e_i)=a_i \in \mathbb K$, existe uma única transformação linear $T: \mathbb K^n \to \mathbb K$ dada por $T(x_1, \dots, x_n) = \sum_{i=1}^n a_i x_i$.

    Essas são todas as transformações lineares de $\mathbb K^n$ em $\mathbb K$.
\end{example}

Quem são todas as transformações lineares de $\mathbb K^n$ em $\mathbb K^m$?

\begin{example}
    Seja $T: \mathbb K^n \to \mathbb K^m$ uma transformação linear.
    Temos que $T(x_1, \dots, x_n)= \sum_{i=1}^n x_i T(e_j)$, onde $e_i$ é o $i$-ésimo vetor da base canônica de $\mathbb K^n$.

    Logo, para qualquer escolha $T(e_j)=(a_{1j}, a_{2j}, \dots, a_{mj}) \in \mathbb K^m$, existe uma única transformação linear $T: \mathbb K^n \to \mathbb K^m$ dada por

    \begin{equation*}
        T(x_1, \dots, x_n) = \left(\sum_{j=1}^n a_{1j} x_j, \sum_{j=1}^n a_{2j} x_j, \dots, \sum_{j=1}^n a_{mj} x_j\right).
    \end{equation*}

    Note que, colocando os vetores em formato de coluna, temos que as coordenadas de $T(x_1, \dots, x_n)$ são dadas pela seguinte coluna:

    \begin{equation*}
        \begin{pmatrix}
            a_{11} & a_{12} & \cdots & a_{1n}\\
            a_{21} & a_{22} & \cdots & a_{2n}\\
            \vdots & \vdots & \ddots & \vdots\\
            a_{m1} & a_{m2} & \cdots & a_{mn}
        \end{pmatrix}
        \begin{pmatrix}
            x_1\\
            x_2\\
            \vdots\\
            x_n
        \end{pmatrix}.
    \end{equation*}
\end{example}

Abaixo, estenderemos esse raciocínio para transformações lineares entre espaços vetoriais quaisquer de dimensão finita positiva.

\begin{proposition}
Sejam $V$ e $W$ espaços vetoriais sobre um corpo $\mathbb K$, com $1\leq \dim V=n<\infty$ e $1\leq \dim W=m<\infty$.
Sejam $\mathcal B = (v_1, \dots, v_n)$ uma base de $V$ e $\mathcal C = (w_1, \dots, w_m)$ uma base de $W$.
\end{proposition}