\section{Noções de Conjuntos}
Utilizaremos, a notação usual de teoria dos conjuntos ao longo deste texto.

Assumiremos, implicitamente, e sem qualquer aprofundamento desnecessário, que estamos trabalhando na \emph{Teoria dos Conjuntos de Zermelo-Fraenkel com o Axioma da Escolha (ZFC)}.
O conhecimento sobre o que se trata tal teoria não é, em nenhum aspecto, necessário para a compreensão deste texto.
Porém, espera-se o entendimento básico sobre o que é a relação de um conjunto, a relação de pertinência $\in$, a relação de subconjunto $\subseteq$, o entendimento de operações básicas como união, interseção, e diferença de conjuntos, a compreensão de notações como $\{a, b, c\}$, da noção de par ordenado, $(a, b)$.

Também espera-se a compreensão das noções básicas de funções, como o entendimento da notação $f: A \to B$, da noção de domínio e imagem de uma função, injetividade, sobrejetividade, bijetividade, invertibilidade de funções e de contradomínios.

As noções de \emph{operações} em um conjunto aparecerão naturalmente ao longo do texto.
Em particular, estabelece-se a seguir o que chamaremos de \emph{operação unária} e de \emph{operação binária}.

Lembramos que $A\times B$ denota o produto cartesiano de dois conjuntos $A$ e $B$, ou seja, o conjunto de todos os pares ordenados $(a, b)$ com $a \in A$ e $b \in B$.

\begin{definition}[Operações]
    Uma \emph{operação unária} em um conjunto $A$ é uma função $*: A \to A$.

    Ao trabalhar com uma operação unária, frequentemente utilizamos notações especiais. Abrevia-se $*(a)$ como $*a$.

    Uma \emph{operação binária} em um conjunto $A$ é uma função $*: A \times A \to A$.

    Ao trabalhar com uma operação binária, frequentemente utilizamos notações especiais. Abrevia-se $*(a, b)$ como $a * b$.
\end{definition}

Por exemplo: ao lidar-se com números reais, a \emph{soma usual} $+: \mathbb{R} \times \mathbb{R} \to \mathbb{R}$ é uma operação binária, de modo que $+(a, b)=a+b$.

Da mesma forma, a operação de \emph{oposto} $-: \mathbb{R} \to \mathbb{R}$ é uma operação unária, de modo que $-(a)= -a$.





