\section{Anéis e corpos}

Anéis e corpos possuem como objetivo generalizar estruturas numéricas conhecidas que possuem uma soma e um produto, como os números inteiros, racionais, reais e complexos.

Se $R$ é um conjunto, uma \textbf{operação binária} em $R$ é uma função $*: R \times R \to R$ 
\begin{definition}
Um \emph{anel} \index{anel} é um conjunto $R$ munido de duas operações binárias, geralmente denotadas por $+$ e $\cdot$, e por dois elementos destacados, geralmente denotados por $0$ e $1$, tais que, para todos $a, b, c \in R$, valem:
\begin{itemize}
    \item[(A1)] $(a+b)+c = a+(b+c)$ (associatividade da soma),
    \item[(A2)] $a+b = b+a$ (comutatividade da soma),
    \item[(A3)] $a+0=a$ (elemento neutro),
    \item[(A4)] existe $x \in R$ tal que $a+x=0$ (elemento oposto),
    \item[(M1)] $(a\cdot b)\cdot c = a\cdot(b\cdot c)$ (associatividade do produto),
    \item[(M2)] $a\cdot 1=1\cdot a=a$ (elemento neutro do produto),
    \item[(D1)] $a\cdot(b+c)=a\cdot b+a\cdot c$ (distributiva à esquerda),
    \item[(D2)] $(a+b)\cdot c=a\cdot c+b\cdot c$ (distributiva à direita).
\end{itemize}

Se, além disso, para todos $a, b \in R$ vale que:
\begin{itemize}
    \item[(M3)] $a\cdot b = b\cdot a$ (comutatividade do produto),
\end{itemize}
então dizemos que $R$ é um \emph{anel comutativo}\index{anel!comutativo}.
\end{definition}

No contexto de anéis, quando não há confusão, é usual denotar $a\cdot b$ simplesmnete por $ab$.
Além disso, valem as regras usuais de precedência de parênteses, de modo que $a+b\cdot c$ é interpretado como $a+(b\cdot c)$, e \textbf{não} como $(a+b)\cdot c$.

Exemplos de anéis incluem os inteiros $\mathbb{Z}$, os números racionais $\mathbb{Q}$, os números reais $\mathbb{R}$ e os números complexos $\mathbb{C}$.

Outro exemplo de anel é o conjunto unitário $\{a\}$ com as operações $a+a=0$ e $a\cdot a=a$, onde $a$ é qualquer objeto. Denotando-se $1=0=a$, verifica-se facilmente que $\{a\}$ é um anel comutativo.
Os anéis dessa forma são chamados de \emph{anéis triviais}\index{anel!trivial}.

Os números naturais $\mathbb{N}$ não formam um anel, pois não possuem elementos opostos para a soma.

Vejamos algumas propriedades básicas.

\begin{proposition}[Unicidade do elemento neutro da soma]
    Seja $R$ um anel. Então só existe um elemento neutro para a soma.
    Ou seja, se $x \in R$ é tal que para todos $a \in R$ vale $a+x=a$, então $x=0$.
\end{proposition}
\begin{proof}
    Como $0$ é elemento neutro da soma, temos que $0+x=x$.
    Por outro lado, como $x$ é elemento neutro da soma, temos que $x+0=x$.
    Logo, $x=0$.
\end{proof}

Do mesmo modo, prova-se a unicidade do elemento neutro do produto.
Deixamos a redação da demonstração como exercício de fixação ao leitor. 

\begin{proposition}[Unicidade do elemento neutro do produto]\label{proposition:unicidadeNeutroProduto}
    Seja $R$ um anel.
    Então só existe um elemento neutro para o produto.
    Ou seja, se $x \in R$ é tal que para todos $a \in R$ vale $a\cdot x=x\cdot a=a$, então $x=1$.
\end{proposition}
\begin{proof}
    Exercício.
\end{proof}



Da mesma forma, os \emph{opostos aditivos} são únicos.

\begin{proposition}[Unicidade dos opostos aditivos]
    Seja $R$ um anel.
    Então, para cada $a \in R$, existe um \emph{único} elemento $x \in R$ tal que $a+x=0$.
\end{proposition}
\begin{proof}
    A existência de um tal $x$ é garantida pela propriedade A4.
    Para a unicidade, suponha que $x$ e $y$ são elementos de $R$ tais que $a+x=0$ e $a+y=0$.
    Segue que:
    \begin{equation*}
        x=x+0=x+(a+y)=(x+a)+y=0+y=y.
    \end{equation*}
\end{proof}

Assim, podemos dar um nome especial para o oposto aditivo de um elemento $a \in R$:

\begin{definition}[Opostos aditivos]\index{oposto aditivo!em um anel}
    Seja $R$ um anel.
    Então, para cada $a \in R$ o \emph{oposto aditivo} de $a$, denotado por $-a$, é o único elemento de $R$ que satisfaz $a+(-a)=0$.

    Em particular, veja que $-0=0$, pois $0+0=0$.
\end{definition}


Uma propriedade particular interessante do elemento nulo é que ele anula, via multiplicação, qualquer outro elemento.

Notacionalmente, é útil definir uma notação binária para a \emph{diferença} de dois elementos.

\begin{definition}[Diferença]
    Seja $R$ um anel e $a, b \in R$.
    A \emph{diferença} de $a$ e $b$, denotada por $a-b$, é definida como $a+(-b)$.
\end{definition}
Algumas propriedades dos elementos opostos são:
\begin{proposition}
    Seja $R$ um anel e $a, b \in R$.
    Então:
    \begin{enumerate}[label=(\roman*)]
        \item $-(-a)=a$.
        \item $-(a+b)=(-a)+(-b)$.
        \item $-0=0$.
        \item $a\cdot 0=0=0\cdot a$.
        \item $(-a)b=a(-b)=-(ab)$.
        \item $(-a)(-b)=ab$.
        \item $-a=(-1)a$.
    \end{enumerate}
\end{proposition}
\begin{proof}
        \begin{enumerate}[label=(\roman*)]
        \item Como $(-a)+a=0$, temos que $-(-a)=a$.
        \item Como $(a+b)+((-a)+(-b))=a+(-a)+b+(-b)=0+0=0$, temos que $-(a+b)=(-a)+(-b)$.
        \item Note que $0+0=0$, temos que $-0=0$.
        \item Veja que $0\cdot a=(0+0)\cdot a=0\cdot a+0\cdot a$.
        Assim, $0=(0\cdot a)+(-(0\cdot a))=[(0\cdot a)+(0\cdot a)]+(-(0\cdot a))=(0\cdot a)+[(0\cdot a)+(-(0\cdot a))]=0\cdot a$.

        Analogamente, $a\cdot 0=0$.
        \item Note que $ab+(-a)b=[a+(-a)]b=0b=0$.
        Assim, $(-a)b=-(ab)$.
        Da mesma forma, $a(-b)=-(ab)$.
        \item Pelo item anterior, temos que $(-a)(-b)=-[a(-b)]=-(-(ab))=ab$.
        \item Finalmente, $(-1)a=-(1\cdot a)=-a$.
    \end{enumerate}
\end{proof}

Assim, as chamadas ``regras de sinais'' decorrem das definições iniciais sobre anéis, e as operações numéricas de subtração usuais podem ser vistas como casos particulares de somas, fazendo-se uso da noção de opostos.

Nesse contexto, como podemos encarar a divisão?
Para isso, precisamos da noção de \emph{inverso multiplicativo}.

\begin{definition}\index{elemento!invertível}
    Seja $R$ um anel e $a \in R$.
    Dizemos que $a$ é \emph{invertível} se existe $b \in R$ tal que $a\cdot b=b\cdot a=1$.
\end{definition}

É de conhecimento do leitor que não existe um elemento $b \in \mathbb{Z}$ tal que $2\cdot b=1$.
Assim, $2$ não é invertível em $\mathbb{Z}$.
Por outro lado, $2$ é invertível em $\mathbb{Q}$, pois $2\cdot \frac{1}{2}=\frac{1}{2}\cdot 2=1$.

Similarmente ao que ocorre com opostos aditivos, os inversos multiplicativos, caso existam, são únicos.

\begin{proposition}
    Seja $R$ um anel e $a \in R$ um elemento invertível.
    Então existe um \emph{único} elemento $b \in R$ tal que $a\cdot b=b\cdot a=1$.
\end{proposition}
\begin{proof}
    A existência de $b$ é garantida pela definição de elemento invertível.
    Para a unicidade, suponha que $b$ e $c$ são elementos de $R$ tais que $a\cdot b=b\cdot a=1$ e $a\cdot c=c\cdot a=1$.

    Segue que $b=b.1=b(ac)=(ba)c=1\cdot c=c$.
\end{proof}

\begin{definition}
\index{inverso multiplicativo}
    Seja $R$ um anel e $a \in R$ um elemento invertível.
    O \emph{inverso multiplicativo} de $a$, denotado por $a^{-1}$, é o único elemento de $R$ tal que $a\cdot a^{-1}=a^{-1}\cdot a=1$.
\end{definition}

Algumas propriedades dos inversos multiplicativos são:

\begin{proposition}
    Seja $R$ um anel e $a, b \in R$ elementos invertíveis. Então:
    \begin{enumerate}[label=(\roman*)]
        \item $ab$ é invertível e $(ab)^{-1}=b^{-1}a^{-1}$.
        \item $a^{-1}$ é invertível e $(a^{-1})^{-1}=a$.
        \item $1$ é invertível e $1^{-1}=1$.
    \end{enumerate}
\end{proposition}

\begin{proof}
    \begin{enumerate}[label=(\roman*)]
        \item Notemos que $(ab)(b^{-1}a^{-1})=abb^{-1}a^{-1}=aa^{-1}=1$. Similarmente, $(b^{-1}a^{-1})(ab)=b^{-1}(a^{-1}a)b=b^{-1}b=1$. Assim, $ab$ é invertível e $(ab)^{-1}=b^{-1}a^{-1}$.
        \item Note que $a^{-1}a=aa^{-1}=1$, logo $a^{-1}$ é invertível e $(a^{-1})^{-1}=a$.
        \item Como $1\cdot 1=1$, temos que $1^{-1}=1$.
    \end{enumerate}
\end{proof}

Finalmente, chegamos a noção de corpo.
Um corpo é um anel comutativo no qual todo elemento é invertível, a excessão do $0$, que deve necessariamente ser diferente de $1$.

\begin{definition}\index{corpo}
    Um \emph{corpo} é um anel comutativo $\mathbb K$ no qual valem, adicionalmente, as seguintes propriedades:
    \begin{itemize}
        \item[(NT)] $0\neq 1$ (não trivialidade).
        \item[(M4)] Para todo $a\in \mathbb K$, se $a\neq 0$ então $a$ é invertível (existência de inversos multiplicativos).
    \end{itemize}
\end{definition}

Assim, em um corpo, todo elemento não nulo é invertível.
De fato, esses são todos, como mostra a proposição a seguir.

\begin{proposition}
    Seja $\mathbb K$ um corpo.
    Então $0\in K$ não é invertível.
\end{proposition}
\begin{proof}
    Suponha por absurdo que $0$ é invertível.
    Então existe $x \in \mathbb K$ tal que $0x=1$.
    Porém, conforme visto, $0x=0$, logo, $0=1$, violando a não trivialidade do corpo.
\end{proof}

Neste texto, nossos principais exemplos de corpos serão o conjunto dos números reais e o conjunto dos números complexos, e, por vezes, o conjunto dos números racionais, com os quais supomos que o leitor tenha alguma familiaridade, ainda que informal.
Porém, como exemplo de um corpo diferente destes, temos:

\begin{example}\label{example:corpoF2}
    Seja $\mathbb F_2=\{0, 1\}$, com as operações de soma e produto definidas por:
    \begin{itemize}
        \item $0+0=0$, $0+1=1$, $1+0=1$, $1+1=0$ (soma módulo 2),
        \item $0\cdot 0=0$, $0\cdot 1=0$, $1\cdot 0=0$, $1\cdot 1=1$ (produto usual).
    \end{itemize}
    Verifica-se que $\mathbb F_2$ é um corpo (ver Exercício~\ref{exercise:corpoF2}).
\end{example}