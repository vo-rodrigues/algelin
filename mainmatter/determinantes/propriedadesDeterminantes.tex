  
\section{Propriedades dos determinantes}
\begin{theorem}[Binet]
    Sejam $A, B \in M_n(R)$.
    Então $\det(AB) = \det(A)\det(B)$.
\end{theorem}
\begin{proof}
Fixe $A$. Defina $f: (R^n)^n \to R$ por $f(v_1, v_2, \ldots, v_n) = \det(Av_1, Av_2, \ldots, Av_n)$.
Note que $f$ é multilinear e alternada e que $f(e_1, e_2, \ldots, e_n) = \det(A)$.
Logo, $f(v_1, v_2, \ldots, v_n) = \det(A) \det(v_1, v_2, \ldots, v_n)$.

Por outro lado, $g:(R^n)^n \to R$ definida por $g(v_1, v_2, \ldots, v_n) = \det(A)\det(v_1, v_2, \ldots, v_n)$ também é multilinear e alternada e vale $\det(A)$ em $(e_1, e_2, \ldots, e_n)$.
Logo, $f(v_1, v_2, \ldots, v_n) = g(v_1, v_2, \ldots, v_n)$ para todos $v_1, v_2, \ldots, v_n \in R^n$.
Em particular, para as colunas de $B$, temos:
\begin{equation*}
    \det(AB) = f(\text{col}_1(B), \text{col}_2(B), \ldots, \text{col}_n(B)) = g(\text{col}_1(B), \text{col}_2(B), \ldots, \text{col}_n(B)) = \det(A)\det(B).
\end{equation*}
\end{proof}
\begin{proposition}
    Seja $R$ um anel comutativo e $A \in M_n(R)$.
    Então $\det(A)=\det(A^t)$, onde $t$ é a matriz transporta de $A$.
\end{proposition}
\begin{proof}
Seja $A=(a_{ij})_{i,j}$.
Temos que $A^t=(b_{ij})_{i, j}=(a_{ji})_{i,j}$.
Logo,
\begin{equation*}
    \begin{aligned}
        \det(A^t) &= \sum_{\sigma \in S_n} \sgn(\sigma) b_{\sigma(1)1} b_{\sigma(2)2} \cdots b_{\sigma(n)n}\\
        &= \sum_{\sigma \in S_n} \sgn(\sigma) \prod_{i=1}^n a_{i\sigma(i)}.
    \end{aligned}
\end{equation*}
Em cada produtório, fazendo $j=\sigma(i)$, temos: que $i=\sigma^{-1}(j)$.
Logo,
\begin{equation*}
    \prod_{i=1}^n a_{i\sigma(i)} = \prod_{j=1}^n a_{\sigma^{-1}(j)j}.
\end{equation*}

Assim:
\begin{equation*}
    \begin{aligned}
        \det(A^t) &= \sum_{\sigma \in S_n} \sgn(\sigma) \prod_{j=1}^n a_{\sigma^{-1}(j)j}.
    \end{aligned}
\end{equation*}
No somatório, fazendo $\rho=\sigma^{-1}$, temos que $\sigma=\rho^{-1}$.

Assim:

\begin{equation*}
    \begin{aligned}
        \det(A^t) &= \sum_{\rho \in S_n} \sgn(\rho^{-1}) \prod_{j=1}^n a_{\rho(j)j}\\
        &= \sum_{\rho \in S_n} \sgn(\rho) \prod_{j=1}^n a_{\rho(j)j}\\
        &= \det(A).
    \end{aligned}
\end{equation*}

Para essa penúltima igualdade, devemos ver que $\sgn(\rho^{-1}) = \sgn(\rho)$.
S $\tau_1, \tau_2, \ldots, \tau_k$ são transposições tais que $\tau_k \circ \cdots \circ \tau_1 = \rho$, então $\tau_1 \circ \cdots \circ \tau_k = \rho^{-1}$.
Do fato que a inversa de uma transposição é ela mesma, temos que $\tau_1 \circ \cdots \circ \tau_k = \rho^{-1}$.
Logo, um número de transposições necessárias para obter $\rho^{-1}$ é o mesmo que para obter $\rho$.
Portanto, $\sgn(\rho^{-1}) = \sgn(\rho)$.
\end{proof}

    \begin{proposition}[Expansão de Laplace em uma coluna]
    Para $n\geq 1$, $A \in M_{n+1}(R)$, e $j \in \{1, \ldots, n+1\}$, temos que
    \begin{equation*}
        \det(A) = \sum_{j=1}^{n+1} (-1)^{j+i} a_{ji} \det(A(i|j)),
    \end{equation*}

    em que $A(i|j)$ é a matriz obtida de $A$ ao remover a linha $i$ e a coluna $j$.
\end{proposition}

\begin{proof}
    Como visto, $\det(A) = \det(A^t)$.
    Seja $\bar i=j$.
    Escreva $A^t=(b_{ij})_{i,j}$ e $A=(a_{ij})_{i,j}$.
    Temos que $b_{ij} = a_{ji}$ para todos $i, j$.
    Usando a expansão de Laplace em na linha $\bar i$ de $A^t$, temos:
    \begin{equation*}
        \det(A)=\det(A^t) = \sum_{k=1}^{n+1} (-1)^{\bar i + k} b_{\bar i k} \det((A^t)(\bar i|k)).
    \end{equation*}
    Note que $(A^t)(\bar i|k) = (A(k|\bar i))^t$.
    Logo, $\det((A^t)(\bar i|k)) = \det(A(k|\bar i))$.

    Assim:

    \begin{equation*}
        \det(A) = \sum_{k=1}^{n+1} (-1)^{\bar i + k} a_{k\bar i} \det(A(k|\bar i)).
    \end{equation*}

    Lembrando que $j=\bar i$, e trocando a letra muda $k$ por $i$ na expressão acima, temos o resultado desejado.
\end{proof}

\begin{example}[Regra de Sarrus]
    Seja $R$ um anel comutativo e $A \in M_3(R)$ dada por
    \begin{equation*}
        A = \begin{pmatrix}
        a_{11} & a_{12} & a_{13}\\
        a_{21} & a_{22} & a_{23}\\
        a_{31} & a_{32} & a_{33}
        \end{pmatrix}.
    \end{equation*}
    Então:
    \begin{equation*}
        \det(A) = a_{11}a_{22}a_{33} + a_{12}a_{23}a_{31} + a_{13}a_{21}a_{32} - a_{13}a_{22}a_{31} - a_{11}a_{23}a_{32} - a_{12}a_{21}a_{33}.
    \end{equation*}
\end{example}
\begin{proof}
    Fazendo a expansão de Laplace na primeira coluna, temos que:
    \begin{equation*}
        \det(A) = a_{11} \det(A(1|1)) - a_{21} \det(A(2|1)) + a_{31} \det(A(3|1)).
    \end{equation*}

    Ou seja:
    \begin{equation*}
        \begin{aligned}
            \det(A) &= a_{11} \begin{vmatrix}
            a_{22} & a_{23}\\
            a_{32} & a_{33}
            \end{vmatrix} - a_{21} \begin{vmatrix}
            a_{12} & a_{13}\\
            a_{32} & a_{33}
            \end{vmatrix} + a_{31} \begin{vmatrix}
            a_{12} & a_{13}\\
            a_{22} & a_{23}
            \end{vmatrix}\\
            &= a_{11}(a_{22}a_{33} - a_{23}a_{32}) - a_{21}(a_{12}a_{33} - a_{13}a_{32}) + a_{31}(a_{12}a_{23} - a_{13}a_{22})\\
            &= a_{11}a_{22}a_{33} + a_{12}a_{23}a_{31} + a_{13}a_{21}a_{32} - a_{13}a_{22}a_{31} - a_{11}a_{23}a_{32} - a_{12}a_{21}a_{33}.
        \end{aligned}
    \end{equation*}
\end{proof}

\begin{example}
    Considere a matriz $A \in M_3(R)$ dada por
    \begin{equation*}
        A = \begin{pmatrix}
        1 & 2 & 3\\
        0 & 1 & 4\\
        5 & 6 & 0
        \end{pmatrix}.
    \end{equation*}
    Então, pela regra de Sarrus, temos que:
    \begin{equation*}
        \begin{aligned}
            \det(A) &= 1 \cdot 1 \cdot 0 + 2 \cdot 4 \cdot 5 + 3 \cdot 0 \cdot 6 - 3 \cdot 1 \cdot 5 - 1 \cdot 4 \cdot 6 - 2 \cdot 0 \cdot 0\\
            &= 0 + 40 + 0 - 15 - 24 - 0 = 1.
        \end{aligned}
    \end{equation*}
\end{example}

Agora vejamos como calcular o determinante em blocos.

\begin{proposition}
    Seja $R$ um anel comutativo e $n, m \geq 1$.
    Sejam $A \in M_n(R)$, $B \in M_{n\times m}(R)$, $0 \in M_{m\times n}(R)$, e $D \in M_m(R)$.
    \begin{equation*}
        \det\begin{pmatrix}
        A & B\\
        0 & D
        \end{pmatrix} = \det(A)\det(D).
    \end{equation*}

\begin{proof}
    Primeiro, vamos provar que, para todo $A \in M_n(R)$ e $B \in M_{n\times m}(R)$, temos que
    \begin{equation*}
        \det\begin{pmatrix}
        A & B\\
        0 & I_m
        \end{pmatrix} = \det(A).
    \end{equation*}

    Para tanto, seja $f: M_{n+m}(R) \to R$ dada por $f(X) = \det\begin{pmatrix}
    X & B\\
    0 & I_m
    \end{pmatrix}$.
    É fácil ver que $f$ é multilinear e alternada em relação às primeiras $n$ colunas de $X$.
    Além disso, $f(I_n) = \det\begin{pmatrix}
    I_n & B\\
    0 & I_m
    \end{pmatrix} = 1$,
    uma vez que as últimas $m$ linhas podem, via escalonamento, eliminar a matriz $B$ sem alterar o determinante.
    Logo, $f(X) = \det(X)$ para todo $X \in M_n(R)$.

    Agora, fixemos $A \in M_n(R)$ e definimos $g: M_{m}(R) \to R$ por $g(Y) = \det\begin{pmatrix}
A & B\\
0 & Y
\end{pmatrix}$.
É fácil ver que $g$ é multilinear e alternada em relação às últimas $m$ colunas de $Y$.
Além disso, $g(I_m) = \det\begin{pmatrix}
A & B\\
0 & I_m
\end{pmatrix}$.
Logo, $g(Y) = \det(A) \det(Y)$ para todo $Y \in M_m(R)$.
Em particular, para $Y = D$, temos o resultado desejado.
\end{proof}
\end{proposition}