\section{Introdução}

Os determinantes são funções que associam a cada matriz quadrada um escalar, de forma que essa associação capture certas propriedades importantes da matriz.

Para defini-los, pensarmos em uma matriz quadrada $n\times n$ como uma coleção de $n$ vetores em um espaço vetorial de dimensão $n$.

Em nossa discussão inicial, pensaremos no corpo dos números reais.
No corpo dos números reais, o determinante mede o "volume" do paralelogramo (ou paralelepípedo) formado por esses vetores.

Listaremos algumas propriedades que esperamos que o determinante satisfaça. Abaixo, todos os vetores são elementos de $\mathbb R^n$.

\begin{enumerate}[label=\alph*)]
    \item $\det(e_1, e_2, \ldots, e_n) = 1$, em que $e_i$ são os vetores da base canônica, pois este é o paralelogramo unitário.
    \item Temos que $\det(v_1, v_2, \ldots, v_n) = 0$ caso tenhamos dois vetores repetidos, pois o paralelogramo colapsa em um espaço de dimensão menor, resultando em volume zero.
    \item Ao multiplicar um vetor por algum escalar, o determinante é multiplicado por esse escalar.
    Ou seja, para todos os escalares $\alpha \in \mathbb R$ e qualquer coordenada $i$, temos que $\det(v_1, \ldots, \alpha v_i, \ldots, v_n) = \alpha \det(v_1, v_2, \ldots, v_n)$.
    Notemos que isso abre margem para a possibilidade de o determinante ser negativo, o que tem a ver com a orientação do paralelogramo.
    \item Se $v_1, v_2, \ldots, v_n$ são vetores e $v_i'$ é outro vetor, então $\det(v_1, \ldots, v_{i-1}, v_i + v_i', v_{i+1}, \ldots, v_n) = \det(v_1, v_2, \ldots, v_n) + \det(v_1, \ldots, v_{i-1}, v_i', v_{i+1}, \ldots, v_n)$.
    Tal propriedade é esperada, pois o paralelogramo determinado pelos vetores do lado esquerdo da igualdade pode ser visto como uma junção do paralelogramo determinado pelos vetores do primeiro determinante do lado direito com o paralelogramo determinado pelos vetores do segundo determinante do lado direito, o que faz as áreas serem somadas.
    Note que isso também abre margem para volumes negativos, o que será ignorado na discussão inicial, mas que tem a ver, novamente, com a orientação.
\end{enumerate}

As propriedades c) e d) indicam que o determinante é uma função \emph{linear em cada coordenada}.

Olhando para elas isoladamente, é possível definir o seguinte conceito.
Abaixo, note que $(R^n)^n=R^n \times R^n \times \dots \times R^n$ ($n$ vezes).

\begin{definition}
Seja $R$ um anel comutativo.
Uma função $f: (R^n)^n \to R$ é dita \emph{multilinear} se, para cada $i \in \{1, \ldots, n\}$, a função obtida ao fixar todos os argumentos de $f$ exceto o $i$-ésimo é linear.
Formalmente, se para todos $v_1, \dots, v_n, v_j'\in R^n$ e todo $\alpha \in R$, temos:
\begin{itemize}[leftmargin=4.1mm]
    \item A função $f$ abre para soma coordenada-a-coordenada, ou seja:
    \begin{equation*}
    f(v_1, \ldots, v_{j-1}, v_j + v_j', v_{j+1}, \ldots, v_n) = f(v_1, \ldots, v_{j-1}, v_j, v_{j+1}, \ldots, v_n) + f(v_1, \ldots, v_{j-1}, v_j', v_{j+1}, \ldots, v_n).
    \end{equation*}
    \item A função $f$ respeita o produto por escalar em cada coordenada, ou seja:
    \begin{equation*}
    f(v_1, \ldots, v_{j-1}, \alpha v_j, v_{j+1}, \ldots, v_n) = \alpha f(v_1, \ldots, v_{j-1}, v_j, v_{j+1}, \ldots, v_n).
    \end{equation*}
\end{itemize}
\end{definition}

Agora olharemos para a propriedade b).
Na presença da multilinearidade, ela é equivalente a outra propriedade no corpo dos números reais.

\begin{proposition}
Seja $f: (R^n)^n \to R$ uma função multilinear.
Considere as seguintes afirmações:
\begin{enumerate}[label=\alph*)]
    \item ($f$ é antissimétrica) Para quaisquer $v_1, \ldots, v_n \in R^n$ e $i, j \in \{1, \ldots, n\}$ com $i \neq j$, temos que:
    \begin{equation*}
        f(v_1, \ldots, v_n)=-f(v_1, \ldots, v_{i-1}, v_j, v_{i+1}, \ldots, v_{j-1}, v_i, v_{j+1}, \ldots, v_n).
    \end{equation*}
    \item ($f$ é alternada) Para quaisquer $v_1, \ldots, v_n \in R^n$ e $i, j \in \{1, \ldots, n\}$ com $i \neq j$, se $v_i = v_j$, então:
    \begin{equation*}
        f(v_1, \ldots, v_n) = 0.
    \end{equation*}
\end{enumerate}
Então a propriedade (b) implica a propriedade (a).
Além disso, se, em $R$, $2(=1+1)$ é tal que para todo $a \in R$, $2a = 0$ implica $a = 0$ então (a) implica (b).
\end{proposition}

\begin{proof}
Suponha que (b) seja verdadeira.
Temos que:
\begin{equation*}
    \begin{aligned}
        0 &= f(v_1, \ldots, v_{i-1}, v_i + v_j, v_{i+1}, \ldots, v_{j-1}, v_i + v_j, v_{j+1}, \ldots, v_n)\\
        &= f(v_1, \ldots, v_n) + f(v_1, \ldots, v_{i-1}, v_j, v_{i+1}, \ldots, v_n)+ f(v_1, \ldots, v_{i-1}, v_j, v_{i+1}, \ldots, v_n)\\
        &\quad  + f(v_1, \ldots, v_{i-1}, v_j, v_{i+1}, \ldots, v_{j-1}, v_i, v_{j+1}, \ldots, v_n)\\
        &\quad =  f(v_1, \ldots, v_n)+ f(v_1, \ldots, v_{i-1}, v_j, v_{i+1}, \ldots, v_{j-1}, v_i, v_{j+1}, \ldots, v_n).
    \end{aligned}
\end{equation*}

Agora suponha que (a) seja verdadeira, além da hipótese adicional sobre $R$, e sejam dados $v_1, \ldots, v_n \in R^n$ tais que existem $i, j \in \{1, \ldots, n\}$ com $i \neq j$ e $v_i = v_j$.
Então, aplicando (a) e trocando $v_i$ por $v_j$ na expressão de (a), temos:
\begin{equation*}
    f(v_1, \ldots, v_n) = -f(v_1, \ldots, v_n).
\end{equation*}
Logo, $2f(v_1, \ldots, v_n) = 0$, o que implica que $f(v_1, \ldots, v_n) = 0$.
\end{proof}

Algumas propriedades de funções $n$-lineares são:
\begin{proposition}
    Seja $f: (R^n)^n \to R$ uma função $n$-linear.
    Então, para quaisquer $v_1, \ldots, v_n \in R^n$ e $i \in \{1, \ldots, n\}$, se $v_i = 0$, então:
    \begin{equation*}
        f(v_1, \ldots, v_n) = 0.
    \end{equation*}
\end{proposition}
\begin{proof}
    Seja $v_1, \ldots, v_n \in R^n$ tais que $v_i = 0$ para algum $i$.
    Então:
    \begin{equation*}
        \begin{aligned}
            f(v_1, \ldots, v_n) &= f(v_1, \ldots, v_{i-1}, 0 + 0, v_{i+1}, \ldots, v_n)\\
            &= f(v_1, \ldots, v_{i-1}, 0, v_{i+1}, \ldots, v_n) + f(v_1, \ldots, v_{i-1}, 0, v_{i+1}, \ldots, v_n)\\
            &= 2 f(v_1, \ldots, v_{i-1}, 0, v_{i+1}, \ldots, v_n).
        \end{aligned}
    \end{equation*}
    Pela hipótese sobre $R$, temos que $f(v_1, \ldots, v_n) = 0$.
\end{proof}
\begin{proposition}
    Seja $f: (R^n)^n \to R$ uma função $n$-linear e alternada.
    Então, para quaisquer $v_1, \ldots, v_n \in R^n$ e $i, j \in \{1, \ldots, n\}$ com $i \neq j$, se $r \in R$, então:
    \begin{equation*}
        f(v_1, \ldots, v_{i-1}, v_i + r v_j, v_{i+1}, \ldots, v_n) = f(v_1, \ldots, v_n).
        \end{equation*}

\end{proposition}
\begin{proof}
    Seja $v_1, \ldots, v_n \in R^n$ e $i, j \in \{1, \ldots, n\}$ com $i \neq j$.
    Então:
    \begin{equation*}
        \begin{aligned}
            f(v_1, \ldots, v_{i-1}, v_i + r v_j, v_{i+1}, \ldots, v_n) &= f(v_1, \ldots, v_n) + r f(v_1, \ldots, v_{i-1}, v_j, v_{i+1}, \ldots, v_n)\\
            &= f(v_1, \ldots, v_n) + r \cdot 0\\
            &= f(v_1, \ldots, v_n).
        \end{aligned}
    \end{equation*}
\end{proof}

Na definição de determinante, o que usaremos é a propriedade de $f$ ser alternada.
Estamos prontos para enunciar a definição de determinante.

\begin{definition}
Seja $R$ um anel comutativo.
O \emph{determinante} $n\times n$ de $R$ é a única função $\det: (R^n)^n \to R$ multilinear, alternada, e que vale $1$ em $(e_1, e_2, \ldots, e_n)$, em que $e_i$ são os vetores da base canônica de $R^n$.
\end{definition}

É claro que, por enquanto, não provamos a existência nem a unicidade de tal função.
Isso será feito na seção a seguir.
