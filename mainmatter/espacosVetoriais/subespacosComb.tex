\section{Subespaços gerados e combinações lineares}

\begin{definition}
    Seja $V$ um espaço vetorial sobre um corpo $\mathbb K$ e $A \subseteq V$.
    O \emph{subespaço gerado} por $A$, denotado por $\langle A \rangle$, é o menor subespaço de $V$ que contém $A$.
    Mais precisamente, $\langle A\rangle$ é o único subespaço $W$ de $V$ que satisfaz:
    \begin{enumerate}[label=(\roman*)]
        \item $A \subseteq W$;
        \item se $U$ é um subespaço de $V$ tal que $A \subseteq U$, então $W \subseteq U$.
    \end{enumerate}
\end{definition}
É claro que devemos ver que o subespaço gerado por $A$ está bem definido, ou seja, que tal subespaço existe e é único.
\begin{lemma}
    Seja $V$ um espaço vetorial sobre um corpo $\mathbb K$ e $A \subseteq V$.
    Então existe um único subespaço de $V$ que satisfaz (i) e (ii) da definição anterior.
\end{lemma}
\begin{proof}
    Começaremos provando a existência.
    Seja $\mathcal S$ a coleção de todos os subespaços de $V$ que contêm $A$, ou seja:
    \begin{equation*}
        \mathcal S = \{W \subseteq V : W \text{ é um subespaço de } V \text{ e } A \subseteq W\}.
    \end{equation*}
    Note que $\mathcal S$ é não vazia, pois $V \in \mathcal S$.
    Seja $W = \bigcap_{U \in \mathcal S} U$.
    Pela Proposição~\ref{prop:intersecaoSubespacos}, $W$ é um subespaço de $V$.
    Além disso, $A \subseteq W=\bigcap_{U \in \mathcal S} U$, pois $A \subseteq U$ para todo $U \in \mathcal S$.
    Assim, (i) é satisfeita.

    Para (ii), seja $U'$ um subespaço de $V$ tal que $A \subseteq U'$.
    Então $U' \in \mathcal S$ e, portanto, $W = \bigcap_{U \in \mathcal S} U \subseteq U'$.

    Agora provaremos a unicidade.
    Suponha que $W_1$ e $W_2$ são subespaços de $V$ que satisfazem (i) e (ii).

    Por (i) aplicado à $W_1$, temos que $A \subseteq W_1$.
    Assim, por (ii) aplicado à $W_2$, com $U=W_1$, temos que $W_2 \subseteq W_1$.
    Analogamente, por (i) aplicado à $W_2$, temos que $A \subseteq W_2$.
    Assim, por (ii) aplicado à $W_1$, com $U=W_2$, temos que $W_1 \subseteq W_2$.

    Portanto, $W_1 = W_2$.
\end{proof}

Convém darmos uma descrição mais construtiva e visualizável do subespaço gerado por um conjunto $A \subseteq V$.
Para isso, introduziremos o conceito de combinação linear.

\begin{definition}
    Seja $V$ um espaço vetorial sobre um corpo $\mathbb K$ e $A \subseteq V$.
    Uma \emph{combinação linear} de elementos de $A$ é um vetor da forma
    \begin{equation*}
        \sum_{i<n} \alpha_i v_i
    \end{equation*}
    onde $(\alpha_n)_{i<n}$ é uma sequência finita de escalares em $\mathbb K$ e $(v_i)_{i<n}$ é uma sequência finita de vetores em $A$ dois-a-dois distintos.
\end{definition}

Na notação acima, adotamos a convenção de que, se $n=0$, a soma vazia é igual ao vetor nulo, $0$.
Dessa forma, $0$ é uma combinação linear de elementos de $A$ qualquer que seja $A$, mesmo que $A$ seja o conjunto vazio.

\begin{proposition}
    Seja $V$ um espaço vetorial sobre um corpo $\mathbb K$ e $A \subseteq V$.
    Então $\langle A\rangle$ é o conjunto de todas as combinações lineares de elementos de $A$.
    Ou seja:

    \begin{equation*}
        \langle A \rangle = \left\{\sum_{i<n} \alpha_i v_i : n \in \mathbb N, (\alpha_i)_{i<n} \text{ é uma sequência finita em } \mathbb K, (v_i)_{i<n} \text{ é uma sequência finita em } A\right\}.
    \end{equation*}
\end{proposition}
\begin{proof}
    Seja $W=\{\sum_{i<n} \alpha_i v_i : n \in \mathbb N, (\alpha_i)_{i<n}$ é uma sequência finita em $\mathbb K$, $(v_i)_{i<n}$ é uma sequência finita injetora em $A\}.$

    Vejamos primeiro que $W\subseteq \langle A\rangle$.
    Procedemos por indução em $n$ para mostrar que dada uma sequência finita injetora $(v_i)_{i<n}$ em $A$ e uma sequência finita $(\alpha_i)_{i<n}$ em $\mathbb K$, $\sum_{i<n} \alpha_i v_i \in \langle A\rangle$.
    \begin{itemize} 
    \item Se $n=0$ temos que $\sum_{i<n} \alpha_i v_i = 0 \in \langle A\rangle$.
    \item Para o passo indutivo, temos que $\sum_{i<n+1} \alpha_i v_i = \alpha_n v_n + \sum_{i<n} \alpha_i v_i$.
    Como $v_n \in \langle A\rangle$, $\alpha_n \in \mathbb K$ e $\sum_{i<n} \alpha_i v_i \in \langle A\rangle$ pela hipótese indutiva, segue que $\sum_{i<n+1} \alpha_i v_i \in \langle A\rangle$.
    \end{itemize}

    Reciprocamente, para ver que $\langle A\rangle \subseteq W$, veremos que $W$ é um subespaço de $V$ que contém $A$.
    De fato, $0 \in W$.
    Dados $u, v \in W$, completando a sequência de escalares com $0$ quando necessário, podemos escrever $u = \sum_{i<n} \alpha_i v_i$ e $v = \sum_{i<n} \beta_i v_i$, onde $(v_i)_{i<n}$ é uma sequência finita injetora em $A$ e $(\alpha_i)_{i<n}$, $(\beta_i)_{i<n}$ são sequências finitas em $\mathbb K$.
    Assim, para $\gamma \in \mathbb K$, temos que
    \begin{equation*}
        \gamma u + v = \sum_{i<n} \gamma \alpha_i v_i + \sum_{i<n} \beta_i v_i = \sum_{i<n} (\gamma \alpha_i + \beta_i) v_i \in W.
    \end{equation*}

    Para ver que $A \subseteq W$, note que, para $v \in A$, temos que $v = 1 \cdot v \in W$.
\end{proof}

Um caso particular importante é o seguinte:

\begin{corollary}
    Seja $V$ um espaço vetorial sobre um corpo $\mathbb K$ e $A=\{v_1, \dots, v_n\}$ um subconjunto finito de $V$, em que $v_i\neq v_j$ para $i \neq j$.
    Então \begin{equation*}
        \langle A \rangle = \left\{\sum_{i=1}^n \alpha_i v_i : (\alpha_i)_{i=1}^n \text{ é uma sequência finita em } \mathbb K\right\}.
    \end{equation*}
\end{corollary}