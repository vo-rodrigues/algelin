\section{Coordenadas}
Nessa seção, introduziremos coordenadas relativas a uma base de um espaço vetorial.

\begin{definition}
    Seja $V$ um espaço vetorial de dimensão finita $n>0$.
    Uma \textbf{base ordenada} de $V$ é uma sequência $\mathcal B = (v_1, v_2, \ldots, v_n)$ tal que $\{v_1, v_2, \ldots, v_n\}$ é uma base de $V$.
\end{definition}

Note que, em particular, se $\mathcal B=(v_1, v_2, \ldots, v_n)$ é uma base ordenada de $V$, então $v_i\neq v_j$ se $i \neq j$, caso contrário a imagem de $\mathcal B$, que é uma base de $V$, teria menos do que $n=\dim V$ elementos.

\begin{proposition}
    Seja $V$ um espaço vetorial de dimensão finita $n>0$ e $\mathcal B = (v_1, v_2, \ldots, v_n)$ uma sequência de elementos de $V$.
    Então, $\mathcal B$ é uma base ordenada de $V$ se, e somente se, para todo $v \in V$, existem um único $(\alpha_1, \dots, \alpha_n)\in \mathbb K^n$ tal que
    \begin{equation*}
        v = \alpha_1 v_1 + \alpha_2 v_2 + \cdots + \alpha_n v_n.
    \end{equation*}
\end{proposition}
\begin{proof}
    Suponha que $\mathcal B$ é uma base ordenada de $V$.
    Como a imagem de $\mathcal B$ gera $V$, existem escalares $\alpha_1, \ldots, \alpha_n$ tais que $v = \alpha_1 v_1 + \alpha_2 v_2 + \cdots + \alpha_n v_n$.
    Suponha que existem $(\beta_1, \ldots, \beta_n)$ tais que $v = \beta_1 v_1 + \beta_2 v_2 + \cdots + \beta_n v_n$.
    Então:
    \begin{equation*}
        0 = v - v = (\alpha_1 - \beta_1)v_1 + (\alpha_2 - \beta_2)v_2 + \cdots + (\alpha_n - \beta_n)v_n.
    \end{equation*}
    Como a imagem de $\mathcal B$ é LI, temos que $\alpha_i - \beta_i = 0$ para todo $i$, ou seja, $\alpha_i = \beta_i$ para todo $i$.
    Assim, $(\alpha_1, \ldots, \alpha_n)=(\beta_1, \ldots, \beta_n)$.

    Reciprocamente, suponha que, para todo $v \in V$, existe um único $(\alpha_1, \dots, \alpha_n)\in \mathbb K^n$ tal que $v = \alpha_1 v_1 + \cdots + \alpha_n v_n$.
    É imediato que a imagem de $\mathcal B$ gera $V$.
    Temos que $v_i\neq v_j$ se $i\ne j$, caso contrário, para $v=v_i$, teríamos duas representações distintas de $v$.
    Finalmente, seja $(\alpha_1, \ldots, \alpha_n)$ tal que $\sum_{i=1}^n \alpha_i v_i = 0$.
    Então, como $0 \in V$, pela hipótese, $(\alpha_1, \ldots, \alpha_n) = (0, \ldots, 0)$.
    Assim, a imagem de $\mathcal B$ é LI.
\end{proof}

\begin{definition}
    Seja $V$ um espaço vetorial de dimensão finita $n>0$ e $\mathcal B = (v_1, \ldots, v_n)$ uma base ordenada de $V$.
    
    Define-se, para cada $v \in V$, o \textbf{vetor de coordenadas} de $v$ em relação a $\mathcal B$, denotado por $(v)_{\mathcal B}$, como o vetor $(\alpha_1, \ldots, \alpha_n) \in \mathbb K^n$ tal que $v = \alpha_1 v_1 + \cdots + \alpha_n v_n$.
\end{definition}

\begin{proposition}
    Seja $V$ um espaço vetorial de dimensão finita $n>0$ e $\mathcal B = (v_1, \ldots, v_n)$ uma base ordenada de $V$.
    Então para todos $u, v \in V$ e $\alpha \in \mathbb K$, temos:
    \begin{enumerate}[label=(\roman*)]
        \item $(u + v)_{\mathcal B} = (u)_{\mathcal B} + (v)_{\mathcal B}$.
        \item $(\alpha v)_{\mathcal B} = \alpha (v)_{\mathcal B}$.
    \end{enumerate}
\end{proposition}
\begin{proof}
    Seja $(u)_{\mathcal B} = (\alpha_1, \ldots, \alpha_n)$ e $(v)_{\mathcal B} = (\beta_1, \ldots, \beta_n)$.
    Então:
    \begin{equation*}
        u + v = \sum_{i=1}^n \alpha_i v_i + \sum_{i=1}^n \beta_i v_i = \sum_{i=1}^n (\alpha_i + \beta_i)v_i,
    \end{equation*}
    logo, $(u + v)_{\mathcal B} = (\alpha_1 + \beta_1, \ldots, \alpha_n + \beta_n) = (u)_{\mathcal B} + (v)_{\mathcal B}$.

    Além disso:
    \begin{equation*}
        \alpha v = \alpha\sum_{i=1}^n \beta_i v_i = \sum_{i=1}^n (\alpha\beta_i)v_i,
    \end{equation*}
    logo, $(\alpha v)_{\mathcal B} = (\alpha\beta_1, \ldots, \alpha\beta_n) = \alpha (v)_{\mathcal B}$.
\end{proof}gi