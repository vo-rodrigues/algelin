\section{Dependência linear e Bases}

\begin{definition}[Dependência linear]
    Seja $V$ um espaço vetorial sobre um corpo $\mathbb K$ e $A \subseteq V$.
    Dizemos que $A$ é \emph{linearmente independente} se, para toda sequência finita injetora $(v_i)_{i<n}$ em $A$ e toda sequência finita $(\alpha_i)_{i<n}$ em $\mathbb K$, a seguinte implicação é verdadeira:
    \begin{equation*}
        \sum_{i<n} \alpha_i v_i = 0 \implies \alpha_i = 0 \text{ para todo } i<n.
    \end{equation*}
    Caso contrário, dizemos que $A$ é \emph{linearmente dependente}.
\end{definition}

\begin{definition}[Base]
    Seja $V$ um espaço vetorial sobre um corpo $\mathbb K$.
    Uma \emph{base} de $V$ é um conjunto $B \subseteq V$ que é linearmente independente e gera $V$, ou seja, $\langle B \rangle = V$.
\end{definition}

\begin{lemma}
    Seja $V$ um espaço vetorial sobre um corpo $\mathbb K$ e $A \subseteq V$.
    Se $v \in V$ é tal que $v \in \langle A\rangle\setminus A$, então $A\cup \{v\}$ é linearmente dependente.
\end{lemma}
\begin{proof}
    Seja $v \in \langle A\rangle\setminus A$.
    Então existem $n \in \mathbb N$, uma sequência finita $(\alpha_i)_{i<n}$ em $\mathbb K$ e uma sequência finita injetora $(v_i)_{i<n}$ em $A$ tais que $v = \sum_{i<n} \alpha_i v_i$.
    Assim, $0 = v - v = \sum_{i<n} \alpha_i v_i + (-1)v$.
    Note que $-1 \neq 0$ e $v \notin A$, logo $A \cup \{v\}$ é linearmente dependente.
\end{proof}

\begin{lemma}
    Seja $V$ um espaço vetorial sobre um corpo $\mathbb K$ e $A \subseteq V$.
    Se $A$ é linearmente independente e $v \in V\setminus A$, então $A \cup \{v\}$ é linearmente independente se, e somente se, $v \notin \langle A \rangle$.
\end{lemma}
\begin{proof}
    Seja $v \in V\setminus A$.
    
    Pelo lema anterior, se $v \in \langle A \rangle$, então $A \cup \{v\}$ é linearmente dependente.

    Reciprocamente, se $A\cup\{v\}$ é linearmente dependente, existem $n \in \mathbb N$, uma sequência finita $(\alpha_i)_{i<n}$ em $\mathbb K$, não toda nula, e uma sequência finita injetora $(u_i)_{i<n}$ em $A \cup \{v\}$ tais que $\sum_{i<n} \alpha_i u_i = 0$.

    Como $A$ é linearmente independente, $v$ deve aparecer em $(u_i)_{i<n}$.
    Assim, em realidade, existem $m \in \mathbb N$, uma sequência finita $(\beta_i)_{i<m}$ em $\mathbb K$, não toda nula, e uma sequência finita injetora $(w_i)_{i<m}$ em $A$ tais que $\sum_{i<m} \beta_i w_i + \gamma v = 0$, com $\gamma \neq 0$.

    Novamente, como $A$ é linearmente independente, $\gamma \neq 0$.
    Assim, $v = -\frac{1}{\gamma} \sum_{i<m} \beta_i w_i \in \langle A \rangle$.
\end{proof}